\documentclass[12pt]{article}


\title{Plotting Points}
\author{FOSSEE}
\usepackage{listings}
\lstset{language=Python,
    basicstyle=\ttfamily,
  commentstyle=\itshape\bfseries,
  showstringspaces=false,
}
\newcommand{\typ}[1]{\lstinline{#1}}
\usepackage[english]{babel}
\usepackage[latin1]{inputenc}
\usepackage{times}
\usepackage[T1]{fontenc}
\usepackage{ae,aecompl}
\usepackage{mathpazo,courier,euler}
\usepackage[scaled=.95]{helvet}

\begin{document}
\date{}
\vspace{-1in}
\begin{center}
\LARGE{Plotting Points}\\
\large{FOSSEE}
\end{center}

\section{Plotting Points with Lists}

\begin{lstlisting}
In []: x = [0, 1, 2, 3]
In []: y = [7, 11, 15, 19]
In []: plot(x, y)
In []: clf()
In []: plot(x, y, 'o')  # Plotting Circles
\end{lstlisting}

\subsection{Line style/marker}
\begin{lstlisting}
The following format string characters are accepted 
to control the line style or marker:
    
    ================    ===============================
    character           description
    ================    ===============================
    '-'                 solid line style
    '--'                dashed line style
    '-.'                dash-dot line style
    ':'                 dotted line style
    '.'                 point marker
    ','                 pixel marker
    'o'                 circle marker
    'v'                 triangle_down marker
    '^'                 triangle_up marker
    '<'                 triangle_left marker
    '>'                 triangle_right marker
    '1'                 tri_down marker
    '2'                 tri_up marker
    '3'                 tri_left marker
    '4'                 tri_right marker
    's'                 square marker
    'p'                 pentagon marker
    '*'                 star marker
    'h'                 hexagon1 marker
    'H'                 hexagon2 marker
    '+'                 plus marker
    'x'                 x marker
    'D'                 diamond marker
    'd'                 thin_diamond marker
    '|'                 vline marker
    '_'                 hline marker
    ================    ===============================

\end{lstlisting}

\subsection{Marker combinations}
\typ{In []: plot(x, y, 'ro')} \\
This plots figure with red colored filled circles.\\
Similarly other combination of colors and marker can be used.
\section{Lists}

Initializing
  \begin{lstlisting}
In []: mtlist = []      # Empty List
In []: lst = [ 1, 2, 3, 4, 5] 
  \end{lstlisting}
Slicing
\begin{lstlisting}
In []: lst[1:3]         # A slice.
Out[]: [2, 3]

In []: lst[1:-1]
Out[]: [2, 3, 4]
\end{lstlisting}
\subsection{Appending to lists}
\begin{lstlisting}
In []: a = [ 6, 7, 8, 9]
In []: b = lst + a
In []: b
Out[]: [1, 2, 3, 4, 5, 6, 7, 8, 9]

In []: lst.append(6)
In []: lst
Out[]: [ 1, 2, 3, 4, 5, 6]
\end{lstlisting}
\subsection{Iterating over a List}
\begin{lstlisting}
In []: for element in b:  # Iterating over the list, element-wise
 ....:     print element       # Print each element
 ....:
\end{lstlisting}

\section{Strings}
\subsection{Splitting Strings}
\begin{lstlisting}
In []: greet = ``hello world''
In []: print greet.split()
Out[]: ['hello', 'world']
In []: greet = ``hello, world''
In []: print greet.split(',')
Out[]: ['hello', ' world'] # Note the whitespace before 'world'
\end{lstlisting}
A string can be split based on the delimiter specified within quotes. A combination of more than one delimiter can also be used.\\
\typ{In []: greet.split(', ')}\\
\typ{Out[]: ['hello', 'world']}\\Note the whitespace is not there anymore.

\section{Plotting from Files}
\subsection{Opening, reading and writing files}

\begin{lstlisting}
  In []: f = open('datafile.txt') #By default opens in read mode. If file does not exist then it throws an exception
  In []: f = open('datafile.txt','r') #Specifying the read mode
  In []: f = open('datafile.txt', 'w') #Opens the file in write mode. If the file already exists, then it deletes all the previous content and opens.
\end{lstlisting}
\subsection{Plotting}
\begin{lstlisting}
l = []
t = []
for line in open('pendulum.txt'):
    point = line.split()
    l.append(float(point[0]))
    t.append(float(point[1]))
tsq = []
for time in t:
    tsq.append(time*time)
plot(l, tsq, '.')
\end{lstlisting}

\end{document}

