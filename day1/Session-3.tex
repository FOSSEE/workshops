%%%%%%%%%%%%%%%%%%%%%%%%%%%%%%%%%%%%%%%%%%%%%%%%%%%%%%%%%%%%%%%%%%%%%%%%%%%%%%%%
% Tutorial slides on Python.
%
% Author: Prabhu Ramachandran <prabhu at aero.iitb.ac.in>
% Copyright (c) 2005-2008, Prabhu Ramachandran
%%%%%%%%%%%%%%%%%%%%%%%%%%%%%%%%%%%%%%%%%%%%%%%%%%%%%%%%%%%%%%%%%%%%%%%%%%%%%%%%

\documentclass[14pt,compress]{beamer}
%\documentclass[draft]{beamer}
%\documentclass[compress,handout]{beamer}
%\usepackage{pgfpages} 
%\pgfpagesuselayout{2 on 1}[a4paper,border shrink=5mm]

% Modified from: generic-ornate-15min-45min.de.tex
\mode<presentation>
{
  \usetheme{Warsaw}
  \useoutertheme{split}
  \setbeamercovered{transparent}
}

\usepackage[english]{babel}
\usepackage[latin1]{inputenc}
%\usepackage{times}
\usepackage[T1]{fontenc}

% Taken from Fernando's slides.
\usepackage{ae,aecompl}
\usepackage{mathpazo,courier,euler}
\usepackage[scaled=.95]{helvet}

\definecolor{darkgreen}{rgb}{0,0.5,0}

\usepackage{listings}
\lstset{language=Python,
    basicstyle=\ttfamily,
    commentstyle=\color{red}\itshape,
  stringstyle=\color{darkgreen},
  showstringspaces=false,
  keywordstyle=\color{blue}\bfseries}

%%%%%%%%%%%%%%%%%%%%%%%%%%%%%%%%%%%%%%%%%%%%%%%%%%%%%%%%%%%%%%%%%%%%%%
% Macros
\setbeamercolor{emphbar}{bg=blue!20, fg=black}
\newcommand{\emphbar}[1]
{\begin{beamercolorbox}[rounded=true]{emphbar} 
      {#1}
 \end{beamercolorbox}
}
\newcounter{time}
\setcounter{time}{0}
\newcommand{\inctime}[1]{\addtocounter{time}{#1}{\tiny \thetime\ m}}

\newcommand{\typ}[1]{\texttt{#1}}

\newcommand{\kwrd}[1]{ \texttt{\textbf{\color{blue}{#1}}}  }

%%% This is from Fernando's setup.
% \usepackage{color}
% \definecolor{orange}{cmyk}{0,0.4,0.8,0.2}
% % Use and configure listings package for nicely formatted code
% \usepackage{listings}
% \lstset{
%    language=Python,
%    basicstyle=\small\ttfamily,
%    commentstyle=\ttfamily\color{blue},
%    stringstyle=\ttfamily\color{orange},
%    showstringspaces=false,
%    breaklines=true,
%    postbreak = \space\dots
% }


%%%%%%%%%%%%%%%%%%%%%%%%%%%%%%%%%%%%%%%%%%%%%%%%%%%%%%%%%%%%%%%%%%%%%%
% Title page
\title[Basic Python]{Python:\\A great programming toolkit}

\author[Asokan \& Prabhu] {Asokan Pichai\\Prabhu Ramachandran}

\institute[IIT Bombay] {Department of Aerospace Engineering\\IIT Bombay}
\date[] {25, July 2009}
%%%%%%%%%%%%%%%%%%%%%%%%%%%%%%%%%%%%%%%%%%%%%%%%%%%%%%%%%%%%%%%%%%%%%%

%\pgfdeclareimage[height=0.75cm]{iitmlogo}{iitmlogo}
%\logo{\pgfuseimage{iitmlogo}}


%% Delete this, if you do not want the table of contents to pop up at
%% the beginning of each subsection:
\AtBeginSubsection[]
{
  \begin{frame}<beamer>
    \frametitle{Outline}
    \tableofcontents[currentsection,currentsubsection]
  \end{frame}
}


% If you wish to uncover everything in a step-wise fashion, uncomment
% the following command: 
%\beamerdefaultoverlayspecification{<+->}

%\includeonlyframes{current,current1,current2,current3,current4,current5,current6}

%%%%%%%%%%%%%%%%%%%%%%%%%%%%%%%%%%%%%%%%%%%%%%%%%%%%%%%%%%%%%%%%%%%%%%
% DOCUMENT STARTS
\begin{document}

\begin{frame}
  \titlepage
\end{frame}

\section{Python}

\begin{frame}
  {Problem set 3}
  As you can guess, idea is to use \kwrd{for}!
\end{frame}

\begin{frame}{Problem 3.1}
  Which of the earlier problems is simpler when we use \kwrd{for} instead of \kwrd{while}? 
\end{frame}

\begin{frame}{Problem 3.2}
  Given an empty chessboard and one Bishop placed in any square, say (r, c), generate the list of all squares the Bishop could move to.
\end{frame}

\begin{frame}[fragile]
  \frametitle{Problem 3.3}

  Given two real numbers \typ{a, b}, and an integer \typ{N}, write a
  function named \typ{linspace( a, b, N)} that returns an ordered list
  of \typ{N} points starting with \typ{a} and ending in \typ{b} and
  equally spaced.\\

  For example, \typ{linspace(0, 5, 11)}, should return, \\
\begin{lstlisting}
[ 0.0 ,  0.5,  1.0 ,  1.5,  2.0 ,  2.5,  
  3.0 ,  3.5,  4.0 ,  4.5,  5.0 ]
\end{lstlisting}
\end{frame}

\begin{frame}[fragile]
  \frametitle{Problem 3.4a (optional)}

Use the \typ{linspace} function and generate a list of N tuples of the form\\
\typ{[($x_1$,f($x_1$)),($x_2$,f($x_2$)),\ldots,($x_N$,f($x_N$))]}\\for the following functions,\begin{itemize}
  \item \typ{f(x) = sin(x)}
  \item \typ{f(x) = sin(x) + sin(10*x)}.
\end{itemize}
\end{frame}

\begin{frame}[fragile]
  \frametitle{Problem 3.4b (optional)}

  Using the tuples generated earlier, determine the intervals where the roots of the functions lie.

  \inctime{15}
\end{frame}

%%%%%%%%%%%%%%%%%%%%%%%%%%%%%%%%%%%%%%%%%%%%%%%%%%%%%%%%%%%%%%%%%%%%%%
% TIME: 15 m, running 185m 
%%%%%%%%%%%%%%%%%%%%%%%%%%%%%%%%%%%%%%%%%%%%%%%%%%%%%%%%%%%%%%%%%%%%%%

\subsection{IO}

\begin{frame}[fragile]
  \frametitle{Simple tokenizing and parsing}
  \begin{lstlisting}
s = """The quick brown fox jumped
       over the lazy dog"""
for word in s.split():
    print word.capitalize()
  \end{lstlisting}
\end{frame}

\begin{frame}[fragile]
  \frametitle{Problem 4.1}
  Given a string like, ``1, 3-7, 12, 15, 18-21'', produce the list \\
  \begin{lstlisting}
    [1,3,4,5,6,7,12,15,18,19,20,21]
  \end{lstlisting}
\end{frame}

\begin{frame}[fragile]
  \frametitle{File handling}
\begin{lstlisting}
>>> f = open('/path/to/file_name')
>>> data = f.read() # Read entire file.
>>> line = f.readline() # Read one line.
>>> f.close() # close the file.
\end{lstlisting}
Writing files
\begin{lstlisting}
>>> f = open('/path/to/file_name', 'w')
>>> f.write('hello world\n')
>>> f.close()
\end{lstlisting}
\begin{itemize}
    \item Everything read or written is a string
\end{itemize}
\emphbar{Try \typ{file?} for more help}
\end{frame}

\begin{frame}[fragile]
    \frametitle{File and \kwrd{for}}
\begin{lstlisting}
>>> f = open('/path/to/file_name')
>>> for line in f:
...     print line
...
\end{lstlisting}
\end{frame}

\begin{frame}{Problem 4.2}
    The given file has lakhs of records in the form:\\
    \typ{RGN;ID;NAME;MARK1;\ldots;MARK5;TOTAL;PFW}\\
    Some entries may be empty.  Read the data from this file and print the
    name of the student with the maximum total marks.
\end{frame}

\begin{frame}{Problem 4.3}
    For the same data file compute the average marks in different
    subjects, the student with the maximum mark in each subject and also
    the standard deviation of the marks.  Do this efficiently.

    \inctime{20}
\end{frame}

%%%%%%%%%%%%%%%%%%%%%%%%%%%%%%%%%%%%%%%%%%%%%%%%%%%%%%%%%%%%%%%%%%%%%%
% TIME: 20 m, running 205m 
%%%%%%%%%%%%%%%%%%%%%%%%%%%%%%%%%%%%%%%%%%%%%%%%%%%%%%%%%%%%%%%%%%%%%%

\subsection{Modules}

\begin{frame}[fragile]
    {Modules}
\begin{lstlisting}
>>> sqrt(2)
Traceback (most recent call last):
  File "<stdin>", line 1, in <module>
NameError: name 'sqrt' is not defined
>>> import math        
>>> math.sqrt(2)
1.4142135623730951
\end{lstlisting}
\end{frame}

\begin{frame}[fragile]
    {Modules}
  \begin{itemize}
    \item The \kwrd{import} keyword ``loads'' a module
    \item One can also use:
      \begin{lstlisting}
>>> from math import sqrt
>>> from math import *
      \end{lstlisting}    
    \item What is the difference?
    \item \alert{Use the later only in interactive mode}
    \end{itemize}
  \emphbar{Package hierarchies}
      \begin{lstlisting}
>>> from os.path import exists
      \end{lstlisting}
\end{frame}

\begin{frame}
  \frametitle{Modules: Standard library}
  \begin{itemize}
  \item Very powerful, ``Batteries included''
  \item Some standard modules:
    \begin{itemize}
    \item Math: \typ{math}, \typ{random}
    \item Internet access: \typ{urllib2}, \typ{smtplib}
    \item System, Command line arguments: \typ{sys}
    \item Operating system interface: \typ{os}
    \item Regular expressions: \typ{re}
    \item Compression: \typ{gzip}, \typ{zipfile}, and \typ{tarfile}
    \item And a whole lot more!
    \end{itemize}
  \item Check out the Python Library reference:
    \url{http://docs.python.org/library/}
  \end{itemize}
\end{frame}

\begin{frame}[fragile]
    {Modules of special interest}
    \begin{description}[matplotlibfor2d]

        \item[\typ{numpy}] Efficient, powerful numeric arrays

        \item[\typ{matplotlib}] Easy, interactive, 2D plotting

        \item[\typ{scipy}] statistics, optimization, integration, linear
            algebra, Fourier transforms, signal and image processing,
            genetic algorithms, ODE solvers, special functions, and more

        \item[Mayavi] Easy, interactive, 3D plotting

    \end{description}
\end{frame}

\begin{frame}[fragile]
    {Creating your own modules}
  \begin{itemize}
  \item Define variables, functions and classes in a file with a
    \typ{.py} extension
  \item This file becomes a module!
  \item Accessible when in the current directory
  \item Use \typ{cd} in IPython to change directory

  \item Naming your module
      \end{itemize}
\end{frame}

\begin{frame}[fragile]
  \frametitle{Modules: example}
  \begin{lstlisting}
# --- arith.py ---
def gcd(a, b):
    if a%b == 0: return b
    return gcd(b, a%b)
def lcm(a, b):
    return a*b/gcd(a, b)
# ------------------
>>> import arith
>>> arith.gcd(26, 65)
13
>>> arith.lcm(26, 65)
130
  \end{lstlisting}
\end{frame}

\begin{frame}[fragile]
  \frametitle{Problem 5.1}

  Put all the functions you have written so far as part of the problems
  into one module called \typ{iitb.py} and use this module from IPython.

\inctime{20}
\end{frame}
%%%%%%%%%%%%%%%%%%%%%%%%%%%%%%%%%%%%%%%%%%%%%%%%%%%%%%%%%%%%%%%%%%%%%%
% TIME: 20 m, running 225m 
%%%%%%%%%%%%%%%%%%%%%%%%%%%%%%%%%%%%%%%%%%%%%%%%%%%%%%%%%%%%%%%%%%%%%%

\subsection{Objects}
\begin{frame}{Objects in Python}
    \begin{itemize}
        \item What is an Object? (Types and classes)
        \item identity
        \item type
        \item method
      \end{itemize}
\end{frame}

\begin{frame}[fragile]
  \frametitle{Why are they useful?}
  \small
  \begin{lstlisting}
for element in (1, 2, 3):
    print element
for key in {'one':1, 'two':2}:
    print key
for char in "123":
    print char
for line in open("myfile.txt"):
    print line
for line in urllib2.urlopen('http://site.com'):
    print line
  \end{lstlisting}
\end{frame}
\begin{frame}{And the winner is \ldots OBJECTS!}
  All objects providing a similar inteface can be used the same way.\\
  Functions (and others) are first-class objects. Can be passed to and returned from functions.
  \inctime{10}
\end{frame}

\subsection{Coding Style in Python}
\begin{frame}{Readability and Consistency}
    \begin{itemize}
        \item Readability Counts!-Code is read more often than its written.
        \item Consistency!
        \item Know when to be inconsistent.
      \end{itemize}
\end{frame}

\begin{frame}[fragile]
  \frametitle{Code Layout}
  \begin{itemize}
        \item Indentation
        \item Tabs or Spaces??
        \item Maximum Line Length
        \item Blank Lines
        \item Encodings
   \end{itemize}
\end{frame}

\begin{frame}{Whitespaces in Expressions}
  \begin{itemize}
        \item When to use extraneous whitespaces??
        \item When to avoid extra whitespaces??
        \item Use one statement per line
   \end{itemize}
\end{frame}

\begin{frame}{Comments}
  \begin{itemize}
        \item No comments better than contradicting comments
        \item Block comments
        \item Inline comments
   \end{itemize}
\end{frame}

\begin{frame}{Docstrings}
  \begin{itemize}
        \item When to write docstrings?
        \item Ending the docstrings
        \item One liner docstrings
   \end{itemize}
\end{frame}
\inctime{10}
\end{document}
