%%%%%%%%%%%%%%%%%%%%%%%%%%%%%%%%%%%%%%%%%%%%%%%%%%%%%%%%%%%%%%%%%%%%%%%%%%%%%%%%
%Tutorial slides on Python.
%
% Author: FOSSEE 
% Copyright (c) 2009, FOSSEE, IIT Bombay
%%%%%%%%%%%%%%%%%%%%%%%%%%%%%%%%%%%%%%%%%%%%%%%%%%%%%%%%%%%%%%%%%%%%%%%%%%%%%%%%

\documentclass[14pt,compress]{beamer}
%\documentclass[draft]{beamer}
%\documentclass[compress,handout]{beamer}
%\usepackage{pgfpages} 
%\pgfpagesuselayout{2 on 1}[a4paper,border shrink=5mm]

% Modified from: generic-ornate-15min-45min.de.tex
\mode<presentation>
{
  \usetheme{Warsaw}
  \useoutertheme{split}
  \setbeamercovered{transparent}
}

\usepackage[english]{babel}
\usepackage[latin1]{inputenc}
%\usepackage{times}
\usepackage[T1]{fontenc}
\usepackage{amsmath}

% Taken from Fernando's slides.
\usepackage{ae,aecompl}
\usepackage{mathpazo,courier,euler}
\usepackage[scaled=.95]{helvet}

\definecolor{darkgreen}{rgb}{0,0.5,0}

\usepackage{listings}
\lstset{language=Python,
    basicstyle=\ttfamily\bfseries,
    commentstyle=\color{red}\itshape,
  stringstyle=\color{darkgreen},
  showstringspaces=false,
  keywordstyle=\color{blue}\bfseries}

%%%%%%%%%%%%%%%%%%%%%%%%%%%%%%%%%%%%%%%%%%%%%%%%%%%%%%%%%%%%%%%%%%%%%%
% Macros
\setbeamercolor{emphbar}{bg=blue!20, fg=black}
\newcommand{\emphbar}[1]
{\begin{beamercolorbox}[rounded=true]{emphbar} 
      {#1}
 \end{beamercolorbox}
}
\newcounter{time}
\setcounter{time}{0}
\newcommand{\inctime}[1]{\addtocounter{time}{#1}{\tiny \thetime\ m}}

\newcommand{\typ}[1]{\lstinline{#1}}

\newcommand{\kwrd}[1]{ \texttt{\textbf{\color{blue}{#1}}}  }

%%% This is from Fernando's setup.
% \usepackage{color}
% \definecolor{orange}{cmyk}{0,0.4,0.8,0.2}
% % Use and configure listings package for nicely formatted code
% \usepackage{listings}
% \lstset{
%    language=Python,
%    basicstyle=\small\ttfamily,
%    commentstyle=\ttfamily\color{blue},
%    stringstyle=\ttfamily\color{orange},
%    showstringspaces=false,
%    breaklines=true,
%    postbreak = \space\dots
% }


%%%%%%%%%%%%%%%%%%%%%%%%%%%%%%%%%%%%%%%%%%%%%%%%%%%%%%%%%%%%%%%%%%%%%%
% Title page
\title[Basic Python]{Matrices, Solution of equations and Integration\\}

\author[FOSSEE] {FOSSEE}

\institute[IIT Bombay] {Department of Aerospace Engineering\\IIT Bombay}
\date[] {31, October 2009\\Day 1, Session 4}
%%%%%%%%%%%%%%%%%%%%%%%%%%%%%%%%%%%%%%%%%%%%%%%%%%%%%%%%%%%%%%%%%%%%%%

%\pgfdeclareimage[height=0.75cm]{iitmlogo}{iitmlogo}
%\logo{\pgfuseimage{iitmlogo}}


%% Delete this, if you do not want the table of contents to pop up at
%% the beginning of each subsection:
\AtBeginSubsection[]
{
  \begin{frame}<beamer>
    \frametitle{Outline}
    \tableofcontents[currentsection,currentsubsection]
  \end{frame}
}

%%\AtBeginSection[]
%%{
  %%\begin{frame}<beamer>
%%    \frametitle{Outline}
  %%  \tableofcontents[currentsection,currentsubsection]
  %%\end{frame}
%%}

% If you wish to uncover everything in a step-wise fashion, uncomment
% the following command: 
%\beamerdefaultoverlayspecification{<+->}

%\includeonlyframes{current,current1,current2,current3,current4,current5,current6}

%%%%%%%%%%%%%%%%%%%%%%%%%%%%%%%%%%%%%%%%%%%%%%%%%%%%%%%%%%%%%%%%%%%%%%
% DOCUMENT STARTS
\begin{document}

\begin{frame}
  \titlepage
\end{frame}

\begin{frame}
  \frametitle{Outline}
  \tableofcontents
%  \pausesections
\end{frame}

\section{Matrices}

\begin{frame}
\frametitle{Matrices: Introduction}
We looked at the Van der Monde matrix in the previous session,\\ 
let us now look at matrices in a little more detail.
\end{frame}

\subsection{Initializing}
\begin{frame}[fragile]
\frametitle{Matrices: Initializing}
\begin{lstlisting}
  In []: a = matrix([[1,2,3],
                     [4,5,6],
                     [7,8,9]])

  In []: a
  Out[]: 
  matrix([[1, 2, 3],
         [4, 5, 6],
         [7, 8, 9]])
\end{lstlisting}
\end{frame}

\subsection{Basic Operations}
\begin{frame}[fragile]
\frametitle{Inverse of a Matrix}

\begin{small}
\begin{lstlisting}
In []: linalg.inv(A)
Out[]: 
matrix([[ 0.07734807,  0.01657459,  0.32044199],
        [ 0.09944751, -0.12154696, -0.01657459],
        [-0.02762431, -0.07734807,  0.17127072]])

\end{lstlisting}
\end{small}
\end{frame}

\begin{frame}[fragile]
\frametitle{Determinant}
\begin{lstlisting}
  In []: linalg.det(a)
  Out[]: -9.5171266700777579e-16
\end{lstlisting}
\end{frame}

\begin{frame}[fragile]
\frametitle{Computing Norms}
\begin{lstlisting}
  In []: linalg.norm(a)
  Out[]: 16.881943016134134
\end{lstlisting}
\end{frame}

\begin{frame}[fragile]
\frametitle{Eigen Values and Eigen Matrix}
\begin{small}
\begin{lstlisting}
  In []: linalg.eigvals(a)
  Out[]: array([1.61168440e+01, -1.11684397e+00, -1.22196337e-15])

  In []: linalg.eig(a)
  Out[]: 
  (array([ 1.61168440e+01, -1.11684397e+00, -1.22196337e-15]),
   matrix([[-0.23197069, -0.78583024,  0.40824829],
          [-0.52532209, -0.08675134, -0.81649658],
          [-0.8186735 ,  0.61232756,  0.40824829]]))
\end{lstlisting}
\end{small}
\end{frame}

\section{Solving linear equations}

\begin{frame}[fragile]
\frametitle{Solution of equations}
Consider,
  \begin{align*}
    3x + 2y - z  & = 1 \\
    2x - 2y + 4z  & = -2 \\
    -x + \frac{1}{2}y -z & = 0
  \end{align*}
Solution:
  \begin{align*}
    x & = 1 \\
    y & = -2 \\
    z & = -2
  \end{align*}
\end{frame}

\begin{frame}[fragile]
\frametitle{Solving using Matrices}
Let us now look at how to solve this using \kwrd{matrices}
  \begin{lstlisting}
    In []: A = matrix([[3,2,-1],[2,-2,4],[-1, 0.5, -1]])
    In []: b = matrix([[1], [-2], [0]])
    In []: x = linalg.solve(A, b)
    In []: Ax = dot(A, x)
    In []: allclose(Ax, b)
    Out[]: True
  \end{lstlisting}
\end{frame}

\begin{frame}[fragile]
\frametitle{Solution:}
\begin{lstlisting}
In []: x
Out[]: 
array([[ 1.],
       [-2.],
       [-2.]])

In []: Ax
Out[]: 
matrix([[  1.00000000e+00],
        [ -2.00000000e+00],
        [  2.22044605e-16]])
\end{lstlisting}
\end{frame}

\section{Integration}

\subsection{Quadrature}

\begin{frame}[fragile]
\frametitle{Quadrature}
\begin{itemize}
\item We wish to find area under a curve
\item Area under $(sin(x) + x^2)$ in $(0,1)$
\item scipy has functions to do that
\end{itemize}
\small{\typ{In []: from scipy.integrate import quad}}
\begin{itemize}
\item Inputs - function to integrate, limits
\end{itemize}
\begin{lstlisting}
In []: x = 0
In []: quad(sin(x)+x**2, 0, 1)
\end{lstlisting}
\alert{\typ{error:}}
\typ{First argument must be a callable function.}
\end{frame}

\begin{frame}[fragile]
\frametitle{Functions - Definition}
\begin{lstlisting}
In []: def f(x):
           return sin(x)+x**2
In []: quad(f, 0, 1)
\end{lstlisting}
\begin{itemize}
\item \typ{def}
\item arguments
\item \typ{return}
\end{itemize}
\end{frame}

\begin{frame}[fragile]
\frametitle{Functions - Calling them}
\begin{lstlisting}
In [15]: f()
---------------------------------------
\end{lstlisting}
\alert{\typ{TypeError:}}\typ{f() takes exactly 1 argument}
\typ{(0 given)}
\begin{lstlisting}
In []: f(0)
Out[]: 0.0
In []: f(1)
Out[]: 1.8414709848078965
\end{lstlisting}
\end{frame}


\begin{frame}[fragile]
\frametitle{Functions - Default Arguments}
\begin{lstlisting}
In []: def f(x=1):
           return sin(x)+x**2
In []: f(10)
Out[]: 99.455978889110625
In []: f(1)
Out[]: 1.8414709848078965
In []: f()
Out[]: 1.8414709848078965
\end{lstlisting}
\end{frame}

\begin{frame}[fragile]
\frametitle{Functions - Keyword Arguments}
\begin{lstlisting}
In []: def f(x=1, y=pi):
           return sin(y)+x**2
In []: f()
Out[]: 1.0000000000000002
In []: f(2)
Out[]: 4.0
In []: f(y=2)
Out[]: 1.9092974268256817
In []: f(y=pi/2,x=0)
Out[]: 1.0
\end{lstlisting}
\end{frame}

\begin{frame}[fragile]
  \frametitle{More on functions}
  \begin{itemize}
  \item Scope of variables in the function is local
  \item Mutable items are \alert{passed by reference}
  \item First line after definition may be a documentation string
    (\alert{recommended!})
  \item Function definition and execution defines a name bound to the
    function
  \item You \emph{can} assign a variable to a function!
  \end{itemize}
\end{frame}

\begin{frame}[fragile]
\frametitle{Quadrature \ldots}
\begin{lstlisting}
In []: quad(f, 0, 1)
\end{lstlisting}
Returns the integral and an estimate of the absolute error in the result.
\begin{itemize}
\item Use \typ{dblquad} for Double integrals
\item Use \typ{tplquad} for Triple integrals
\end{itemize}
\end{frame}

\subsection{ODEs}

\begin{frame}[fragile]
\frametitle{ODE Integration}
We shall use the simple ODE of a simple pendulum. 
\begin{equation*}
\ddot{\theta} = -\frac{g}{L}sin(\theta)
\end{equation*}
\begin{itemize}
\item This equation can be written as a system of two first order ODEs
\end{itemize}
\begin{align}
\dot{\theta} &= \omega \\
\dot{\omega} &= -\frac{g}{L}sin(\theta) \\
 \text{At}\ t &= 0 : \nonumber \\
 \theta = \theta_0\quad & \&\quad  \omega = 0 \nonumber
\end{align}
\end{frame}

\begin{frame}[fragile]
\frametitle{Solving ODEs using SciPy}
\begin{itemize}
\item We use the \typ{odeint} function from scipy to do the integration
\item Define a function as below
\end{itemize}
\begin{lstlisting}
In []: def pend_int(unknown, t, p):
  ....     theta, omega = unknown
  ....     g, L = p
  ....     f=[omega, -(g/L)*sin(theta)]
  ....     return f
  ....
\end{lstlisting}
\end{frame}

\begin{frame}[fragile]
\frametitle{Solving ODEs using SciPy \ldots}
\begin{itemize}
\item \typ{t} is the time variable \\ 
\item \typ{p} has the constants \\
\item \typ{initial} has the initial values
\end{itemize}
\begin{lstlisting}
In []: t = linspace(0, 10, 101)
In []: p=(-9.81, 0.2)
In []: initial = [10*2*pi/360, 0]
\end{lstlisting}
\end{frame}

\begin{frame}[fragile]
\frametitle{Solving ODEs using SciPy \ldots}

\small{\typ{In []: from scipy.integrate import odeint}}
\begin{lstlisting}
In []: pend_sol = odeint(pend_int, 
                         initial,t, 
                         args=(p,))
\end{lstlisting}
\end{frame}

\begin{frame}
  \frametitle{Things we have learned}
  \begin{itemize}
  \item
  \item
  \item Functions
    \begin{itemize}
    \item Definition
    \item Calling
    \item Default Arguments
    \item Keyword Arguments
    \end{itemize}
    \item Integration
    \begin{itemize}
      \item Quadrature
      \item ODEs
    \end{itemize}
  \end{itemize}
\end{frame}

\end{document}

