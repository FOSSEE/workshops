%%%%%%%%%%%%%%%%%%%%%%%%%%%%%%%%%%%%%%%%%%%%%%%%%%%%%%%%%%%%%%%%%%%%%%%%%%%%%%%%
%Tutorial slides on Python.
%
% Author: FOSSEE 
% Copyright (c) 2009, FOSSEE, IIT Bombay
%%%%%%%%%%%%%%%%%%%%%%%%%%%%%%%%%%%%%%%%%%%%%%%%%%%%%%%%%%%%%%%%%%%%%%%%%%%%%%%%

\documentclass[14pt,compress]{beamer}
%\documentclass[draft]{beamer}
%\documentclass[compress,handout]{beamer}
%\usepackage{pgfpages} 
%\pgfpagesuselayout{2 on 1}[a4paper,border shrink=5mm]

% Modified from: generic-ornate-15min-45min.de.tex
\mode<presentation>
{
  \usetheme{Warsaw}
  \useoutertheme{split}
  \setbeamercovered{transparent}
}

\usepackage[english]{babel}
\usepackage[latin1]{inputenc}
%\usepackage{times}
\usepackage[T1]{fontenc}
\usepackage{amsmath}

% Taken from Fernando's slides.
\usepackage{ae,aecompl}
\usepackage{mathpazo,courier,euler}
\usepackage[scaled=.95]{helvet}

\definecolor{darkgreen}{rgb}{0,0.5,0}

\usepackage{listings}
\lstset{language=Python,
    basicstyle=\ttfamily\bfseries,
    commentstyle=\color{red}\itshape,
  stringstyle=\color{darkgreen},
  showstringspaces=false,
  keywordstyle=\color{blue}\bfseries}

%%%%%%%%%%%%%%%%%%%%%%%%%%%%%%%%%%%%%%%%%%%%%%%%%%%%%%%%%%%%%%%%%%%%%%
% Macros
\setbeamercolor{emphbar}{bg=blue!20, fg=black}
\newcommand{\emphbar}[1]
{\begin{beamercolorbox}[rounded=true]{emphbar} 
      {#1}
 \end{beamercolorbox}
}
\newcounter{time}
\setcounter{time}{0}
\newcommand{\inctime}[1]{\addtocounter{time}{#1}{\tiny \thetime\ m}}

\newcommand{\typ}[1]{\lstinline{#1}}

\newcommand{\kwrd}[1]{ \texttt{\textbf{\color{blue}{#1}}}  }

%%% This is from Fernando's setup.
% \usepackage{color}
% \definecolor{orange}{cmyk}{0,0.4,0.8,0.2}
% % Use and configure listings package for nicely formatted code
% \usepackage{listings}
% \lstset{
%    language=Python,
%    basicstyle=\small\ttfamily,
%    commentstyle=\ttfamily\color{blue},
%    stringstyle=\ttfamily\color{orange},
%    showstringspaces=false,
%    breaklines=true,
%    postbreak = \space\dots
% }


%%%%%%%%%%%%%%%%%%%%%%%%%%%%%%%%%%%%%%%%%%%%%%%%%%%%%%%%%%%%%%%%%%%%%%
% Title page
\title[Basic Python]{Matrices, Solution of equations and Integration\\}

\author[FOSSEE] {FOSSEE}

\institute[IIT Bombay] {Department of Aerospace Engineering\\IIT Bombay}
\date[] {31, October 2009\\Day 1, Session 4}
%%%%%%%%%%%%%%%%%%%%%%%%%%%%%%%%%%%%%%%%%%%%%%%%%%%%%%%%%%%%%%%%%%%%%%

%\pgfdeclareimage[height=0.75cm]{iitmlogo}{iitmlogo}
%\logo{\pgfuseimage{iitmlogo}}


%% Delete this, if you do not want the table of contents to pop up at
%% the beginning of each subsection:
\AtBeginSubsection[]
{
  \begin{frame}<beamer>
    \frametitle{Outline}
    \tableofcontents[currentsection,currentsubsection]
  \end{frame}
}

%%\AtBeginSection[]
%%{
  %%\begin{frame}<beamer>
%%    \frametitle{Outline}
  %%  \tableofcontents[currentsection,currentsubsection]
  %%\end{frame}
%%}

% If you wish to uncover everything in a step-wise fashion, uncomment
% the following command: 
%\beamerdefaultoverlayspecification{<+->}

%\includeonlyframes{current,current1,current2,current3,current4,current5,current6}

%%%%%%%%%%%%%%%%%%%%%%%%%%%%%%%%%%%%%%%%%%%%%%%%%%%%%%%%%%%%%%%%%%%%%%
% DOCUMENT STARTS
\begin{document}

\begin{frame}
  \titlepage
\end{frame}

\begin{frame}
  \frametitle{Outline}
  \tableofcontents
%  \pausesections
\end{frame}

\section{Matrices}
\subsection{Initializing}
\begin{frame}[fragile]
\frametitle{Matrices: Initializing}
\begin{lstlisting}
  In []: a = matrix([[1,2,3],
                     [4,5,6],
                     [7,8,9]])

  In []: a
  Out[]: 
  matrix([[1, 2, 3],
         [4, 5, 6],
         [7, 8, 9]])
\end{lstlisting}
\end{frame}

\subsection{Basic Operations}
\begin{frame}[fragile]
\frametitle{Inverse of a Matrix}
\begin{small}
\begin{lstlisting}
  In []: linalg.inv(a)
  Out[]: 
  matrix([[  3.15221191e+15,  -6.30442381e+15,   3.15221191e+15],
          [ -6.30442381e+15,   1.26088476e+16,  -6.30442381e+15],
          [  3.15221191e+15,  -6.30442381e+15,   3.15221191e+15]])
\end{lstlisting}
\end{small}
\end{frame}

\begin{frame}[fragile]
\frametitle{Determinant}
\begin{lstlisting}
  In []: linalg.det(a)
  Out[]: -9.5171266700777579e-16
\end{lstlisting}
\end{frame}

\begin{frame}[fragile]
\frametitle{Computing Norms}
\begin{lstlisting}
  In []: linalg.norm(a)
  Out[]: 16.881943016134134
\end{lstlisting}
\end{frame}

\begin{frame}[fragile]
\frametitle{Eigen Values and Eigen Matrix}
\begin{small}
\begin{lstlisting}
  In []: linalg.eigvals(a)
  Out[]: array([  1.61168440e+01,  -1.11684397e+00,  -1.22196337e-15])

  In []: linalg.eig(a)
  Out[]: 
  (array([  1.61168440e+01,  -1.11684397e+00,  -1.22196337e-15]),
   matrix([[-0.23197069, -0.78583024,  0.40824829],
          [-0.52532209, -0.08675134, -0.81649658],
          [-0.8186735 ,  0.61232756,  0.40824829]]))
\end{lstlisting}
\end{small}
\end{frame}

\section{Solving linear equations}
\begin{frame}[fragile]
\frametitle{Solution of equations}
Example problem: Consider the set of equations
\vspace{-0.1in}
  \begin{align*}
    3x + 2y - z  & = 1 \\
    2x - 2y + 4z  & = -2 \\
    -x + \frac{1}{2}y -z & = 0
  \end{align*}
\vspace{-0.08in}
  To Solve this, 
  \begin{lstlisting}
    In []: A = array([[3,2,-1],[2,-2,4],[-1, 0.5, -1]])
    In []: b = array([1, -2, 0])
    In []: x = linalg.solve(A, b)
    In []: Ax = dot(A, x)
    In []: allclose(Ax, b)
    Out[]: True
  \end{lstlisting}
\end{frame}

\section{Integration}

\subsection{ODEs}

\begin{frame}[fragile]
\frametitle{ODE Integration}
We shall use the simple ODE of a simple pendulum. 
\begin{equation*}
\ddot{\theta} = -\frac{g}{L}sin(\theta)
\end{equation*}
\begin{itemize}
\item This equation can be written as a system of two first order ODEs
\end{itemize}
\begin{align}
\dot{\theta} &= \omega \\
\dot{\omega} &= -\frac{g}{L}sin(\theta) \\
 \text{At}\ t &= 0 : \nonumber \\
 \theta = \theta_0\quad & \&\quad  \omega = 0 \nonumber
\end{align}
\end{frame}

\begin{frame}[fragile]
\frametitle{Solving ODEs using SciPy}
\begin{itemize}
\item We use the \typ{odeint} function from scipy to do the integration
\item Define a function as below
\end{itemize}
\begin{lstlisting}
In []: def pend_int(unknown, t, p):
  ....     theta, omega = unknown
  ....     g, L = p
  ....     f=[omega, -(g/L)*sin(theta)]
  ....     return f
  ....
\end{lstlisting}
\end{frame}

\begin{frame}[fragile]
\frametitle{Solving ODEs using SciPy \ldots}
\begin{itemize}
\item \typ{t} is the time variable \\ 
\item \typ{p} has the constants \\
\item \typ{initial} has the initial values
\end{itemize}
\begin{lstlisting}
In []: t = linspace(0, 10, 101)
In []: p=(-9.81, 0.2)
In []: initial = [10*2*pi/360, 0]
\end{lstlisting}
\end{frame}

\begin{frame}[fragile]
\frametitle{Solving ODEs using SciPy \ldots}

\small{\typ{In []: from scipy.integrate import odeint}}
\begin{lstlisting}
In []: pend_sol = odeint(pend_int, 
                         initial,t, 
                         args=(p,))
\end{lstlisting}
\end{frame}

\subsection{Quadrature}

\begin{frame}[fragile]
\frametitle{Quadrature}
Calculate the area under $(sin(x) + x^2)$ in the range $(0,1)$
\small{\typ{In []: from scipy.integrate import quad}}
  \begin{lstlisting}
In []: f(x):
        return sin(x)+x**2
In []: integrate.quad(f, 0, 1)
  \end{lstlisting}
\end{frame}

\end{document}

