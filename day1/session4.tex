%%%%%%%%%%%%%%%%%%%%%%%%%%%%%%%%%%%%%%%%%%%%%%%%%%%%%%%%%%%%%%%%%%%%%%%%%%%%%%%%
%Tutorial slides on Python.
%
% Author: FOSSEE 
% Copyright (c) 2009, FOSSEE, IIT Bombay
%%%%%%%%%%%%%%%%%%%%%%%%%%%%%%%%%%%%%%%%%%%%%%%%%%%%%%%%%%%%%%%%%%%%%%%%%%%%%%%%

\documentclass[14pt,compress]{beamer}
%\documentclass[draft]{beamer}
%\documentclass[compress,handout]{beamer}
%\usepackage{pgfpages} 
%\pgfpagesuselayout{2 on 1}[a4paper,border shrink=5mm]

% Modified from: generic-ornate-15min-45min.de.tex
\mode<presentation>
{
  \usetheme{Warsaw}
  \useoutertheme{split}
  \setbeamercovered{transparent}
}

\usepackage[english]{babel}
\usepackage[latin1]{inputenc}
%\usepackage{times}
\usepackage[T1]{fontenc}
\usepackage{amsmath}

% Taken from Fernando's slides.
\usepackage{ae,aecompl}
\usepackage{mathpazo,courier,euler}
\usepackage[scaled=.95]{helvet}

\definecolor{darkgreen}{rgb}{0,0.5,0}

\usepackage{listings}
\lstset{language=Python,
    basicstyle=\ttfamily\bfseries,
    commentstyle=\color{red}\itshape,
  stringstyle=\color{darkgreen},
  showstringspaces=false,
  keywordstyle=\color{blue}\bfseries}

%%%%%%%%%%%%%%%%%%%%%%%%%%%%%%%%%%%%%%%%%%%%%%%%%%%%%%%%%%%%%%%%%%%%%%
% Macros
\setbeamercolor{emphbar}{bg=blue!20, fg=black}
\newcommand{\emphbar}[1]
{\begin{beamercolorbox}[rounded=true]{emphbar} 
      {#1}
 \end{beamercolorbox}
}
\newcounter{time}
\setcounter{time}{0}
\newcommand{\inctime}[1]{\addtocounter{time}{#1}{\tiny \thetime\ m}}

\newcommand{\typ}[1]{\lstinline{#1}}

\newcommand{\kwrd}[1]{ \texttt{\textbf{\color{blue}{#1}}}  }

%%% This is from Fernando's setup.
% \usepackage{color}
% \definecolor{orange}{cmyk}{0,0.4,0.8,0.2}
% % Use and configure listings package for nicely formatted code
% \usepackage{listings}
% \lstset{
%    language=Python,
%    basicstyle=\small\ttfamily,
%    commentstyle=\ttfamily\color{blue},
%    stringstyle=\ttfamily\color{orange},
%    showstringspaces=false,
%    breaklines=true,
%    postbreak = \space\dots
% }


%%%%%%%%%%%%%%%%%%%%%%%%%%%%%%%%%%%%%%%%%%%%%%%%%%%%%%%%%%%%%%%%%%%%%%
% Title page
\title[Basic Python]{Matrices, Solution of equations}

\author[FOSSEE] {FOSSEE}

\institute[IIT Bombay] {Department of Aerospace Engineering\\IIT Bombay}
\date[] {31, October 2009\\Day 1, Session 4}
%%%%%%%%%%%%%%%%%%%%%%%%%%%%%%%%%%%%%%%%%%%%%%%%%%%%%%%%%%%%%%%%%%%%%%

%\pgfdeclareimage[height=0.75cm]{iitmlogo}{iitmlogo}
%\logo{\pgfuseimage{iitmlogo}}


%% Delete this, if you do not want the table of contents to pop up at
%% the beginning of each subsection:
\AtBeginSubsection[]
{
  \begin{frame}<beamer>
    \frametitle{Outline}
    \tableofcontents[currentsection,currentsubsection]
  \end{frame}
}

\AtBeginSection[]
{
  \begin{frame}<beamer>
    \frametitle{Outline}
    \tableofcontents[currentsection,currentsubsection]
  \end{frame}
}

% If you wish to uncover everything in a step-wise fashion, uncomment
% the following command: 
%\beamerdefaultoverlayspecification{<+->}

%\includeonlyframes{current,current1,current2,current3,current4,current5,current6}

%%%%%%%%%%%%%%%%%%%%%%%%%%%%%%%%%%%%%%%%%%%%%%%%%%%%%%%%%%%%%%%%%%%%%%
% DOCUMENT STARTS
\begin{document}

\begin{frame}
  \titlepage
\end{frame}

\begin{frame}
  \frametitle{Outline}
  \tableofcontents
%  \pausesections
\end{frame}

\section{Matrices}

\begin{frame}
\frametitle{Matrices: Introduction}
We looked at the Van der Monde matrix in the previous session,\\ 
let us now look at matrices in a little more detail.
\end{frame}

\subsection{Initializing}
\begin{frame}[fragile]
\frametitle{Matrices: Initializing}
\begin{lstlisting}
In []: A = matrix([[ 1,  1,  2, -1],
                   [ 2,  5, -1, -9],
                   [ 2,  1, -1,  3],
                   [ 1, -3,  2,  7]])
In []: A
Out[]: 
matrix([[ 1,  1,  2, -1],
        [ 2,  5, -1, -9],
        [ 2,  1, -1,  3],
        [ 1, -3,  2,  7]])
\end{lstlisting}
\end{frame}

\subsection{Basic Operations}

\begin{frame}[fragile]
\frametitle{Transpose of a Matrix}
\begin{lstlisting}
In []: linalg.transpose(A)
Out[]:
matrix([[ 1,  2,  2,  1],
        [ 1,  5,  1, -3],
        [ 2, -1, -1,  2],
        [-1, -9,  3,  7]])
\end{lstlisting}
\end{frame}

\begin{frame}[fragile]
  \frametitle{Sum of all elements}
  \begin{lstlisting}
In []: linalg.sum(A)
Out[]: 12
  \end{lstlisting}
\end{frame}

\begin{frame}[fragile]
  \frametitle{Matrix Addition}
  \begin{lstlisting}
In []: B = matrix([[3,2,-1,5],
                   [2,-2,4,9],
                   [-1,0.5,-1,-7],
                   [9,-5,7,3]])
In []: linalg.add(A,B)
Out[]: 
matrix([[  4. ,   3. ,   1. ,   4. ],
        [  4. ,   3. ,   3. ,   0. ],
        [  1. ,   1.5,  -2. ,  -4. ],
        [ 10. ,  -8. ,   9. ,  10. ]])
  \end{lstlisting}
\end{frame}

\begin{frame}[fragile]
\frametitle{Matrix Multiplication}
\begin{lstlisting}
In []: linalg.multiply(A, B)
Out[]: 
matrix([[  3. ,   2. ,  -2. ,  -5. ],
        [  4. , -10. ,  -4. , -81. ],
        [ -2. ,   0.5,   1. , -21. ],
        [  9. ,  15. ,  14. ,  21. ]])
\end{lstlisting}
\end{frame}

\begin{frame}[fragile]
\frametitle{Inverse of a Matrix}
\begin{small}
\begin{lstlisting}
In []: linalg.inv(A)
Out[]: 
matrix([[-0.5 ,  0.55, -0.15,  0.7 ],
        [ 0.75, -0.5 ,  0.5 , -0.75],
        [ 0.5 , -0.15, -0.05, -0.1 ],
        [ 0.25, -0.25,  0.25, -0.25]])
\end{lstlisting}
\end{small}
\end{frame}

\begin{frame}[fragile]
\frametitle{Determinant}
\begin{lstlisting}
In []: det(A)
Out[66]: 80.0
\end{lstlisting}
\end{frame}

\begin{frame}[fragile]
\frametitle{Eigen Values and Eigen Matrix}
\begin{small}
\begin{lstlisting}
In []: linalg.eig(A)
Out[]: 
(array([ 11.41026155,   3.71581643,  -0.81723144,  -2.30884654]),
 matrix([[ 0.12300187, -0.53899627,  0.63269982,  0.56024583],
        [ 0.8225266 , -0.67562403, -0.63919634, -0.20747251],
        [-0.04763219, -0.47575453, -0.3709497 , -0.80066041],
        [-0.55321941, -0.16331814, -0.23133374,  0.04497415]]))
\end{lstlisting}
\end{small}
\end{frame}

\begin{frame}[fragile]
\frametitle{Computing Norms}
\begin{lstlisting}
In []: linalg.norm(A)
Out[]: 14.0
\end{lstlisting}
\end{frame}

\begin{frame}[fragile]
  \frametitle{Single Value Decomposition}
  \begin{small}
  \begin{lstlisting}
In []: linalg.svd(A)
Out[]: 
(matrix([[-0.08588113,  0.29164297, -0.74892678,  0.58879325],
        [-0.79093255,  0.39530483, -0.11188116, -0.45347812],
        [ 0.1523891 ,  0.78799358,  0.51966138,  0.29290907],
        [ 0.58636823,  0.37113957, -0.39565558, -0.60156827]]),
 array([ 13.17656506,   3.76954116,   2.79959047,   0.57531379]),
 matrix([[-0.05893795, -0.42858358,  0.12442679,  0.89295039],
        [ 0.80364672,  0.51537891,  0.03774111,  0.29514767],
        [-0.11752501,  0.14226922, -0.96333552,  0.19476145],
        [-0.58040171,  0.72832696,  0.23468759,  0.27855956]]))
\end{lstlisting}
  \end{small}
\end{frame}

\section{Solving linear equations}

\begin{frame}[fragile]
\frametitle{Solution of equations}
Consider,
  \begin{align*}
    3x + 2y - z  & = 1 \\
    2x - 2y + 4z  & = -2 \\
    -x + \frac{1}{2}y -z & = 0
  \end{align*}
Solution:
  \begin{align*}
    x & = 1 \\
    y & = -2 \\
    z & = -2
  \end{align*}
\end{frame}

\begin{frame}[fragile]
\frametitle{Solving using Matrices}
Let us now look at how to solve this using \kwrd{matrices}
  \begin{lstlisting}
    In []: A = matrix([[3,2,-1],
                       [2,-2,4],
                       [-1, 0.5, -1]])
    In []: b = matrix([[1], [-2], [0]])
    In []: x = linalg.solve(A, b)
    In []: Ax = dot(A, x)
  \end{lstlisting}
\end{frame}

\begin{frame}[fragile]
\frametitle{Solution:}
\begin{lstlisting}
In []: x
Out[]: 
array([[ 1.],
       [-2.],
       [-2.]])
\end{lstlisting}
\end{frame}

\begin{frame}[fragile]
\frametitle{Let's check!}
\begin{lstlisting}
In []: Ax
Out[]: 
matrix([[  1.00000000e+00],
        [ -2.00000000e+00],
        [  2.22044605e-16]])
\end{lstlisting}
\begin{block}{}
The last term in the matrix is actually \alert{0}!\\
We can use \kwrd{allclose()} to check.
\end{block}
\begin{lstlisting}
In []: allclose(Ax, b)
Out[]: True
\end{lstlisting}
\end{frame}

\subsection{Exercises}

\begin{frame}[fragile]
\frametitle{Problem}

\end{frame}

\begin{frame}[fragile]
\frametitle{Problem}
Solve the set of equations:
\begin{align*}
  x + y + 2z -w & = 3\\
  2x + 5y - z - 9w & = -3\\
  2x + y -z + 3w & = -11 \\
  x - 3y + 2z + 7w & = -5\\
\end{align*}
\end{frame}

\section{Summary}
\begin{frame}
  \frametitle{Summary}
So what did we learn??
  \begin{itemize}
  \item Matrices
    \begin{itemize}
      \item Transpose
      \item Addition
      \item Multiplication
      \item Inverse of a matrix
      \item Determinant
      \item Eigen values and Eigen matrix
      \item Norms
      \item Single Value Decomposition
    \end{itemize}
  \item Solving linear equations
  \end{itemize}
\end{frame}

\end{document}
