%%%%%%%%%%%%%%%%%%%%%%%%%%%%%%%%%%%%%%%%%%%%%%%%%%%%%%%%%%%%%%%%%%%%%%%%%%%%%%%%
% Tutorial slides on Python.
%
% Author: Prabhu Ramachandran <prabhu at aero.iitb.ac.in>
% Copyright (c) 2005-2008, Prabhu Ramachandran
%%%%%%%%%%%%%%%%%%%%%%%%%%%%%%%%%%%%%%%%%%%%%%%%%%%%%%%%%%%%%%%%%%%%%%%%%%%%%%%%

\documentclass[14pt,compress]{beamer}
%\documentclass[draft]{beamer}
%\documentclass[compress,handout]{beamer}
%\usepackage{pgfpages} 
%\pgfpagesuselayout{2 on 1}[a4paper,border shrink=5mm]

% Modified from: generic-ornate-15min-45min.de.tex
\mode<presentation>
{
  \usetheme{Warsaw}
  \useoutertheme{split}
  \setbeamercovered{transparent}
}

\usepackage[english]{babel}
\usepackage[latin1]{inputenc}
%\usepackage{times}
\usepackage[T1]{fontenc}

% Taken from Fernando's slides.
\usepackage{ae,aecompl}
\usepackage{mathpazo,courier,euler}
\usepackage[scaled=.95]{helvet}

\definecolor{darkgreen}{rgb}{0,0.5,0}

\usepackage{listings}
\lstset{language=Python,
    basicstyle=\ttfamily\bfseries,
    commentstyle=\color{red}\itshape,
  stringstyle=\color{darkgreen},
  showstringspaces=false,
  keywordstyle=\color{blue}\bfseries}

%%%%%%%%%%%%%%%%%%%%%%%%%%%%%%%%%%%%%%%%%%%%%%%%%%%%%%%%%%%%%%%%%%%%%%
% Macros
\setbeamercolor{emphbar}{bg=blue!20, fg=black}
\newcommand{\emphbar}[1]
{\begin{beamercolorbox}[rounded=true]{emphbar} 
      {#1}
 \end{beamercolorbox}
}
\newcounter{time}
\setcounter{time}{0}
\newcommand{\inctime}[1]{\addtocounter{time}{#1}{\tiny \thetime\ m}}

\newcommand{\typ}[1]{\texttt{#1}}

\newcommand{\kwrd}[1]{ \texttt{\textbf{\color{blue}{#1}}}  }

%%% This is from Fernando's setup.
% \usepackage{color}
% \definecolor{orange}{cmyk}{0,0.4,0.8,0.2}
% % Use and configure listings package for nicely formatted code
% \usepackage{listings}
% \lstset{
%    language=Python,
%    basicstyle=\small\ttfamily,
%    commentstyle=\ttfamily\color{blue},
%    stringstyle=\ttfamily\color{orange},
%    showstringspaces=false,
%    breaklines=true,
%    postbreak = \space\dots
% }


%%%%%%%%%%%%%%%%%%%%%%%%%%%%%%%%%%%%%%%%%%%%%%%%%%%%%%%%%%%%%%%%%%%%%%
% Title page
\title[Basic Python]{Python:\\A great programming toolkit}

\author[Asokan \& Prabhu] {Asokan Pichai\\Prabhu Ramachandran}

\institute[IIT Bombay] {Department of Aerospace Engineering\\IIT Bombay}
\date[] {10, October 2009}
%%%%%%%%%%%%%%%%%%%%%%%%%%%%%%%%%%%%%%%%%%%%%%%%%%%%%%%%%%%%%%%%%%%%%%

%\pgfdeclareimage[height=0.75cm]{iitmlogo}{iitmlogo}
%\logo{\pgfuseimage{iitmlogo}}


%% Delete this, if you do not want the table of contents to pop up at
%% the beginning of each subsection:
\AtBeginSubsection[]
{
  \begin{frame}<beamer>
    \frametitle{Outline}
    \tableofcontents[currentsection,currentsubsection]
  \end{frame}
}


% If you wish to uncover everything in a step-wise fashion, uncomment
% the following command: 
%\beamerdefaultoverlayspecification{<+->}

%\includeonlyframes{current,current1,current2,current3,current4,current5,current6}

%%%%%%%%%%%%%%%%%%%%%%%%%%%%%%%%%%%%%%%%%%%%%%%%%%%%%%%%%%%%%%%%%%%%%%
% DOCUMENT STARTS
\begin{document}

\begin{frame}
  \titlepage
\end{frame}

\section{Python}

\subsection{Dictionary}
\begin{frame}{Dictionary}
  \begin{itemize}
    \item aka associative arrays, key-value pairs, hashmaps, hashtables \ldots    
    \item \typ{ d = \{ ``Hitchhiker's guide'' : 42, ``Terminator'' : ``I'll be back''\}}
    \item lists and tuples index: 0 \ldots n
    \item dictionaries index using strings
    \item aka key-value pairs
    \item what can be keys?
  \end{itemize}
\end{frame}

\begin{frame}{Dictionary \ldots }
  \begin{itemize}
    \item \alert{Unordered}
      \begin{block}{Standard usage}
        for key in dict:\\
            <use> dict[key] \# => value
      \end{block}
    \item \typ{d.keys()} returns a list
    \item can we have duplicate keys?
  \end{itemize}
\end{frame}

\begin{frame} {Problem Set 2.1}
  \begin{description}
\item[2.1.1] You are given date strings of the form ``29, Jul 2009'', or ``4 January 2008''. In other words a number a string and another number, with a comma sometimes separating the items.Write a function that takes such a string and returns a tuple (yyyy, mm, dd) where all three elements are ints.
    \item[2.1.2] Count word frequencies in a file.
    \item[2.1.3] Find the most used Python keywords in your Python code (import keyword).
\end{description}

\inctime{20}
\end{frame}

\subsection{Set}
\begin{frame}[fragile]
  \frametitle{Set}
    \begin{itemize}
      \item Simplest container, mutable
      \item No ordering, no duplicates
      \item usual suspects: union, intersection, subset \ldots
      \item >, >=, <, <=, in, \ldots
    \end{itemize}
    \begin{lstlisting}
>>> f10 = set([1,2,3,5,8])
>>> p10 = set([2,3,5,7])
>>> f10|p10
set([1, 2, 3, 5, 7, 8])
>>> f10&p10
set([2, 3, 5])
>>> f10-p10
set([8, 1])
\end{lstlisting}
\end{frame}

\begin{frame}[fragile]
  \frametitle{Set}
    \begin{lstlisting}
>>> p10-f10, f10^p10
set([7]), set([1, 7, 8])
>>> set([2,3]) < p10
True
>>> set([2,3]) <= p10
True
>>> 2 in p10
True
>>> 4 in p10
False
>>> len(f10)
5
\end{lstlisting}
\end{frame}


\begin{frame}
  \frametitle{Problem set 2.2}
  \begin{description}
    \item[2.2.1] Given a dictionary of the names of students and their marks, identify how many duplicate marks are there? and what are these?
    \item[2.2.2] Given a string of the form ``4-7, 9, 12, 15'' find the numbers missing in this list for a given range.
\end{description}
\inctime{15}
\end{frame}

\subsection{Functions Reloaded!}
\begin{frame}[fragile]
    \frametitle{Advanced functions}
    \begin{itemize}
        \item default args
        \item var args
        \item keyword args
        \item scope
        \item \typ{global}
      \end{itemize}
\end{frame}

\begin{frame}[fragile]
  \frametitle{Functions: default arguments}
  \small
  \begin{lstlisting}
def ask_ok(prompt, retries=4,
           complaint='Yes or no!'):
    while True:
        ok = raw_input(prompt)
        if ok in ('y', 'ye', 'yes'): 
            return True
        if ok in ('n', 'no', 'nop',
                  'nope'): 
            return False
        retries = retries - 1
        if retries < 0: 
            raise IOError, 'bad user'
        print complaint
  \end{lstlisting}
\end{frame}

\begin{frame}[fragile]
  \frametitle{Functions: keyword arguments}
  \small
  \begin{lstlisting}
def parrot(voltage, state='a stiff', 
           action='voom', type='Royal Blue'):
    print "-- This parrot wouldn't", action,
    print "if you supply", voltage, "Volts."
    print "-- Lovely plumage, the", type
    print "-- It's", state, "!"

parrot(1000)
parrot(action = 'VOOOOOM', voltage = 1000000)
parrot('a thousand',
       state = 'pushing up the daisies')
parrot('a million', 'bereft of life', 'jump')
\end{lstlisting}
\end{frame}

\begin{frame}[fragile]
  \frametitle{Functions: arbitrary argument lists}
  \begin{itemize}
  \item Arbitrary number of arguments using \verb+*args+ or
    \verb+*whatever+
  \item Keyword arguments using \verb+**kw+
  \item Given a tuple/dict how do you call a function?
    \begin{itemize}
    \item Using argument unpacking
    \item For positional arguments: \verb+foo(*[5, 10])+
    \item For keyword args: \verb+foo(**{'a':5, 'b':10})+
    \end{itemize}
  \end{itemize}
\end{frame}

  \begin{frame}[fragile]
\begin{lstlisting}
def foo(a=10, b=100):
    print a, b
def func(*args, **keyword):
    print args, keyword
# Unpacking:
args = [5, 10]
foo(*args)
kw = {'a':5, 'b':10}
foo(**kw)
\end{lstlisting}
    \inctime{15} 
\end{frame}

\subsection{Functional programming}
\begin{frame}[fragile]
    \frametitle{Functional programming}
What is the basic idea?\\
Why is it interesting?\\
\typ{map, reduce, filter}\\
list comprehension\\
generators
    \inctime{10} 
\end{frame}

\subsection{Debugging}
\begin{frame}[fragile]
 \frametitle{Errors}
 \begin{lstlisting}
>>> while True print 'Hello world'
  File "<stdin>", line 1, in ?
    while True print 'Hello world'
                   ^
SyntaxError: invalid syntax
\end{lstlisting}
\end{frame}

\begin{frame}[fragile]
 \frametitle{Exceptions}
 \begin{lstlisting}
>>> print spam
Traceback (most recent call last):
  File "<stdin>", line 1, in <module>
NameError: name 'spam' is not defined

>>> 1 / 0
Traceback (most recent call last):
  File "<stdin>", line 1, in <module>
ZeroDivisionError: integer division 
or modulo by zero
\end{lstlisting}
\end{frame}

\begin{frame}[fragile]
    \frametitle{Debugging effectively}

    \begin{itemize}
        \item  \kwrd{print} based strategy
        \item Process: Hypothesis, test, refine, rinse-repeat
        \item Using \typ{\%debug} and \typ{\%pdb} in IPython
    \end{itemize}
    \inctime{10} 
\end{frame}

\begin{frame}[fragile]
\frametitle{Debugging: example}
\small
\begin{lstlisting}
>>> import pdb
>>> import mymodule
>>> pdb.run('mymodule.test()')
> <string>(1)<module>()
(Pdb) continue
Traceback (most recent call last):
  File "<stdin>", line 1, in <module>
  File "/usr/lib/python2.6/pdb.py", line 1207, in run
    Pdb().run(statement, globals, locals)
  File "/usr/lib/python2.6/bdb.py", line 368, in run
    exec cmd in globals, locals
  File "<string>", line 1, in <module>
  File "mymodule.py", line 2, in test
    print spam
NameError: global name 'spam' is not defined
\end{lstlisting}
\end{frame}

\begin{frame}[fragile]
\frametitle{Debugging in IPython}
\small
\begin{lstlisting}
In [1]: %pdb
Automatic pdb calling has been turned ON
In [2]: import mymodule
In [3]: mymodule.test()
----------------------------------------------
NameError    Traceback (most recent call last)
/media/python/iitb/workshops/day1/<ipython console> in <module>()
/media/python/iitb/workshops/day1/mymodule.pyc in test()
      1 def test():
----> 2     print spam
NameError: global name 'spam' is not defined
> /media/python/iitb/workshops/day1/mymodule.py(2)test()
      0     print spam
ipdb>
\end{lstlisting}
\end{frame}

\begin{frame}[fragile]
\frametitle{Debugging: Exercise}
\end{frame}

\end{document}
