%%%%%%%%%%%%%%%%%%%%%%%%%%%%%%%%%%%%%%%%%%%%%%%%%%%%%%%%%%%%%%%%%%%%%%%%%%%%%%%%
% Tutorial slides on Python.
%
% Author: Prabhu Ramachandran <prabhu at aero.iitb.ac.in>
% Copyright (c) 2005-2009, Prabhu Ramachandran
%%%%%%%%%%%%%%%%%%%%%%%%%%%%%%%%%%%%%%%%%%%%%%%%%%%%%%%%%%%%%%%%%%%%%%%%%%%%%%%%

\documentclass[14pt,compress]{beamer}
%\documentclass[draft]{beamer}
%\documentclass[compress,handout]{beamer}
%\usepackage{pgfpages} 
%\pgfpagesuselayout{2 on 1}[a4paper,border shrink=5mm]

% Modified from: generic-ornate-15min-45min.de.tex
\mode<presentation>
{
  \usetheme{Warsaw}
  \useoutertheme{split}
  \setbeamercovered{transparent}
}

\usepackage[english]{babel}
\usepackage[latin1]{inputenc}
%\usepackage{times}
\usepackage[T1]{fontenc}

% Taken from Fernando's slides.
\usepackage{ae,aecompl}
\usepackage{mathpazo,courier,euler}
\usepackage[scaled=.95]{helvet}

\definecolor{darkgreen}{rgb}{0,0.5,0}

\usepackage{listings}
\lstset{language=Python,
    basicstyle=\ttfamily\bfseries,
    commentstyle=\color{red}\itshape,
  stringstyle=\color{darkgreen},
  showstringspaces=false,
  keywordstyle=\color{blue}\bfseries}

\usepackage{pgf}

%%%%%%%%%%%%%%%%%%%%%%%%%%%%%%%%%%%%%%%%%%%%%%%%%%%%%%%%%%%%%%%%%%%%%%
% Macros
\setbeamercolor{emphbar}{bg=blue!20, fg=black}
\newcommand{\emphbar}[1]
{\begin{beamercolorbox}[rounded=true]{emphbar} 
      {#1}
 \end{beamercolorbox}
}
\newcounter{time}
\setcounter{time}{0}
\newcommand{\inctime}[1]{\addtocounter{time}{#1}{\tiny \thetime\ m}}

\newcommand{\typ}[1]{\texttt{#1}}

\newcommand{\kwrd}[1]{ \texttt{\textbf{\color{blue}{#1}}}  }

%%% This is from Fernando's setup.
% \usepackage{color}
% \definecolor{orange}{cmyk}{0,0.4,0.8,0.2}
% % Use and configure listings package for nicely formatted code
% \usepackage{listings}
% \lstset{
%    language=Python,
%    basicstyle=\small\ttfamily,
%    commentstyle=\ttfamily\color{blue},
%    stringstyle=\ttfamily\color{orange},
%    showstringspaces=false,
%    breaklines=true,
%    postbreak = \space\dots
% }


%%%%%%%%%%%%%%%%%%%%%%%%%%%%%%%%%%%%%%%%%%%%%%%%%%%%%%%%%%%%%%%%%%%%%%
% Title page
\title[Basic Python]{Python:\\Advanced Python data structures, Functions and Debugging}

\author[FOSSEE Team] {Asokan Pichai\\Prabhu Ramachandran}

\institute[IIT Bombay] {Department of Aerospace Engineering\\IIT Bombay}
\date[] {10, October 2009\\Day 1, Session 4}
%%%%%%%%%%%%%%%%%%%%%%%%%%%%%%%%%%%%%%%%%%%%%%%%%%%%%%%%%%%%%%%%%%%%%%

%\pgfdeclareimage[height=0.75cm]{iitmlogo}{iitmlogo}
%\logo{\pgfuseimage{iitmlogo}}


%% Delete this, if you do not want the table of contents to pop up at
%% the beginning of each subsection:
\AtBeginSubsection[]
{
  \begin{frame}<beamer>
    \frametitle{Outline}
    \tableofcontents[currentsection,currentsubsection]
  \end{frame}
}


% If you wish to uncover everything in a step-wise fashion, uncomment
% the following command: 
%\beamerdefaultoverlayspecification{<+->}

%\includeonlyframes{current,current1,current2,current3,current4,current5,current6}

%%%%%%%%%%%%%%%%%%%%%%%%%%%%%%%%%%%%%%%%%%%%%%%%%%%%%%%%%%%%%%%%%%%%%%
% DOCUMENT STARTS
\begin{document}

\begin{frame}
  \titlepage
\end{frame}

\section{Advanced Data structures}

\subsection{Dictionary}
\begin{frame}{Dictionary}
  \begin{itemize}
    \item lists and tuples index: 0 \ldots n
    \item dictionaries index using strings
    \item \typ{ d = \{ ``Hitchhiker's guide'' : 42, ``Terminator'' : ``I'll be back''\}}
    \item \typ{d[``Terminator''] => ``I'll be back''}
    \item aka associative array, key-value pair, hashmap, hashtable \ldots    
    \item what can be keys?
  \end{itemize}
\end{frame}

\begin{frame}{Dictionary \ldots }
  \begin{itemize}
    \item \alert{Unordered}
      \begin{block}{Standard usage}
        for key in dict:\\
        \ \ \ \ print dict[key]
      \end{block}
    \item \typ{d.keys()} returns a list
    \item can we have duplicate keys?
  \end{itemize}
  \inctime{5}
\end{frame}

\begin{frame} {Problem Set 6.1}
  \begin{description}
\item[6.1.1] You are given date strings of the form ``29, Jul 2009'', or ``4 January 2008''. In other words a number a string and another number, with a comma sometimes separating the items.Write a function that takes such a string and returns a tuple (yyyy, mm, dd) where all three elements are ints.
    \item[6.1.2] Count word frequencies in a file.
    \item[6.1.3] Find the most used Python keywords in your Python code (import keyword).
\end{description}

\inctime{10}
\end{frame}

\subsection{Set}
\begin{frame}[fragile]
  \frametitle{Set}
    \begin{itemize}
      \item Simplest container, mutable
      \item No ordering, no duplicates
      \item usual suspects: union, intersection, subset \ldots
      \item >, >=, <, <=, in, \ldots
    \end{itemize}
    \begin{lstlisting}
>>> f10 = set([1,2,3,5,8])
>>> p10 = set([2,3,5,7])
>>> f10|p10
set([1, 2, 3, 5, 7, 8])
>>> f10&p10
set([2, 3, 5])
>>> f10-p10
set([8, 1])
\end{lstlisting}
\end{frame}

\begin{frame}[fragile]
  \frametitle{Set}
    \begin{lstlisting}
>>> p10-f10, f10^p10
set([7]), set([1, 7, 8])
>>> set([2,3]) < p10
True
>>> set([2,3]) <= p10
True
>>> 2 in p10
True
>>> 4 in p10
False
>>> len(f10)
5
\end{lstlisting}
\inctime{5}
\end{frame}

\begin{frame}
  \frametitle{Problem set 6.2}
  \begin{description}
    \item[6.2.1] Given a dictionary of the names of students and their marks, identify how many duplicate marks are there? and what are these?
    \item[6.2.2] Given a string of the form ``4-7, 9, 12, 15'' find the numbers missing in this list for a given range.
\end{description}
\inctime{10}
\end{frame}


\section{Functions Reloaded!}
\begin{frame}[fragile]
    \frametitle{Advanced functions}
    \begin{itemize}
        \item default args
        \item var args
        \item keyword args
        \item scope
        \item \typ{global}
      \end{itemize}
\end{frame}

\subsection{Default arguments}
\begin{frame}[fragile]
  \frametitle{Functions: default arguments}
  \small
  \begin{lstlisting}
def ask_ok(prompt, complaint='Yes or no!'):
    while True:
        ok = raw_input(prompt)
        if ok in ('y', 'ye', 'yes'): 
            return True
        if ok in ('n', 'no', 'nop',
                  'nope'): 
            return False
        print complaint

ask_ok('?')
ask_ok('?', '[Y/N]')
  \end{lstlisting}
\end{frame}

\subsection{Keyword arguments}
\begin{frame}[fragile]
  \frametitle{Functions: keyword arguments}
  \small
  \begin{lstlisting}
def ask_ok(prompt, complaint='Yes or no!'):
    while True:
        ok = raw_input(prompt)
        if ok in ('y', 'ye', 'yes'): 
            return True
        if ok in ('n', 'no', 'nop',
                  'nope'): 
            return False
        print complaint

ask_ok(prompt='?')
ask_ok(prompt='?', complaint='[y/n]')
ask_ok(complaint='[y/n]', prompt='?')
\end{lstlisting}
\inctime{15} 
\end{frame}

\section{Functional programming}
\begin{frame}[fragile]
    \frametitle{Functional programming}
    \begin{itemize}
      \item What is the basic idea?
      \item Why is it interesting?
      \item \typ{map, reduce, filter}
      \item list comprehension
      \item generators
    \end{itemize}
\end{frame}

\subsection{List comprehensions}
\begin{frame}[fragile]
    \frametitle{List Comprehensions}
Lets say we want to squares of all the numbers from 1 to 100
    \begin{lstlisting}
squares = []
for i in range(1, 100):
    squares.append(i * i)
    \end{lstlisting}
    \begin{lstlisting}
# list comprehension
squares = [i*i for i in range(1, 100)]
     \end{lstlisting}
Which is more readable?
\end{frame}

\begin{frame}[fragile]
    \frametitle{List Comprehensions}
What if you had a more complex function?
Lets say we want squares of numbers from 1 to 100 ending in 1, 2, 5, 7 only
    \begin{lstlisting}
squares = []
for i in range(1, 100):
    if i % 10 in [1, 2, 5, 7]:
        squares.append(i * i)
    \end{lstlisting}
    \begin{lstlisting}
# list comprehension
squares = [i*i for i in range(1, 100)
           if i % 10 in [1, 2, 5, 7]]
     \end{lstlisting}
Which is more readable?
\end{frame}

\begin{frame}[fragile]
    \frametitle{map() function}
    map() function accomplishes the same as list comprehensions
    \begin{lstlisting}
>>> def square(x): return x*x
...
>>> map(square, range(1, 100))
[1, 8, 27, 64, 125, 216, 343, 512, 729, 1000]
    \end{lstlisting}
\inctime{15}
\end{frame}

\section{Debugging}
\subsection{Errors and Exceptions}
\begin{frame}[fragile]
 \frametitle{Errors}
 \begin{lstlisting}
>>> while True print 'Hello world'
 \end{lstlisting}
\pause
  \begin{lstlisting}
  File "<stdin>", line 1, in ?
    while True print 'Hello world'
                   ^
SyntaxError: invalid syntax
\end{lstlisting}
\end{frame}

\begin{frame}[fragile]
 \frametitle{Exceptions}
 \begin{lstlisting}
>>> print spam
\end{lstlisting}
\pause
\begin{lstlisting}
Traceback (most recent call last):
  File "<stdin>", line 1, in <module>
NameError: name 'spam' is not defined
\end{lstlisting}
\end{frame}

\begin{frame}[fragile]
 \frametitle{Exceptions}
 \begin{lstlisting}
>>> 1 / 0
\end{lstlisting}
\pause
\begin{lstlisting}
Traceback (most recent call last):
  File "<stdin>", line 1, in <module>
ZeroDivisionError: integer division 
or modulo by zero
\end{lstlisting}
\end{frame}

\subsection{Strategy}
\begin{frame}[fragile]
    \frametitle{Debugging effectively}
    \begin{itemize}
        \item \kwrd{print} based strategy
        \item Process:
    \end{itemize}
\pgfimage[interpolate=true,width=5cm,height=5cm]{DebugginDiagram.png}
\end{frame}

\begin{frame}[fragile]
    \frametitle{Debugging effectively}
    \begin{itemize}
      \item Using \typ{\%debug} in IPython
    \end{itemize}
\end{frame}

\begin{frame}[fragile]
\frametitle{Debugging in IPython}
\small
\begin{lstlisting}
In [1]: import mymodule
In [2]: mymodule.test()
---------------------------------------------
NameError   Traceback (most recent call last)
<ipython console> in <module>()
mymodule.py in test()
      1 def test():
----> 2     print spam
NameError: global name 'spam' is not defined

In [3]: %debug
> mymodule.py(2)test()
      0     print spam
ipdb> 
\end{lstlisting}
\inctime{15} 
\end{frame}

\subsection{Exercise}
\begin{frame}[fragile]
\frametitle{Debugging: Exercise}
\small
\begin{lstlisting}
import keyword
f = open('/path/to/file')

freq = {}
for line in f:
    words = line.split()
    for word in words:
        key = word.strip(',.!;?()[]: ')
        if keyword.iskeyword(key):
            value = freq[key]
            freq[key] = value + 1

print freq
\end{lstlisting}
\inctime{10}
\end{frame}

\begin{frame}
  \frametitle{What did we learn?}
  \begin{itemize}
    \item Dictionaries
    \item Sets
    \item Default and keyword arguments
    \item Functional Programming, list comprehensions
    \item Errors and Exceptions in Python
    \item Debugging: \%debug in IPython
  \end{itemize}
\end{frame}
\end{document}
