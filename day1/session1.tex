%%%%%%%%%%%%%%%%%%%%%%%%%%%%%%%%%%%%%%%%%%%%%%%%%%%%%%%%%%%%%%%%%%%%%%%%%%%%%%%%
%Tutorial slides on Python.
%
% Author: FOSSEE 
% Copyright (c) 2009, FOSSEE, IIT Bombay
%%%%%%%%%%%%%%%%%%%%%%%%%%%%%%%%%%%%%%%%%%%%%%%%%%%%%%%%%%%%%%%%%%%%%%%%%%%%%%%%

\documentclass[14pt,compress]{beamer}
%\documentclass[draft]{beamer}
%\documentclass[compress,handout]{beamer}
%\usepackage{pgfpages} 
%\pgfpagesuselayout{2 on 1}[a4paper,border shrink=5mm]

% Modified from: generic-ornate-15min-45min.de.tex
\mode<presentation>
{
  \usetheme{Warsaw}
  \useoutertheme{split}
  \setbeamercovered{transparent}
}

\usepackage[english]{babel}
\usepackage[latin1]{inputenc}
%\usepackage{times}
\usepackage[T1]{fontenc}

% Taken from Fernando's slides.
\usepackage{ae,aecompl}
\usepackage{mathpazo,courier,euler}
\usepackage[scaled=.95]{helvet}

\definecolor{darkgreen}{rgb}{0,0.5,0}

\usepackage{listings}
\lstset{language=Python,
    basicstyle=\ttfamily\bfseries,
    commentstyle=\color{red}\itshape,
  stringstyle=\color{darkgreen},
  showstringspaces=false,
  keywordstyle=\color{blue}\bfseries}

%%%%%%%%%%%%%%%%%%%%%%%%%%%%%%%%%%%%%%%%%%%%%%%%%%%%%%%%%%%%%%%%%%%%%%
% Macros
\setbeamercolor{emphbar}{bg=blue!20, fg=black}
\newcommand{\emphbar}[1]
{\begin{beamercolorbox}[rounded=true]{emphbar} 
      {#1}
 \end{beamercolorbox}
}
\newcounter{time}
\setcounter{time}{0}
\newcommand{\inctime}[1]{\addtocounter{time}{#1}{\tiny \thetime\ m}}

\newcommand{\typ}[1]{\lstinline{#1}}

\newcommand{\kwrd}[1]{ \texttt{\textbf{\color{blue}{#1}}}  }

%%% This is from Fernando's setup.
% \usepackage{color}
% \definecolor{orange}{cmyk}{0,0.4,0.8,0.2}
% % Use and configure listings package for nicely formatted code
% \usepackage{listings}
% \lstset{
%    language=Python,
%    basicstyle=\small\ttfamily,
%    commentstyle=\ttfamily\color{blue},
%    stringstyle=\ttfamily\color{orange},
%    showstringspaces=false,
%    breaklines=true,
%    postbreak = \space\dots
% }

%%%%%%%%%%%%%%%%%%%%%%%%%%%%%%%%%%%%%%%%%%%%%%%%%%%%%%%%%%%%%%%%%%%%%%
% Title page
\title[Interactive Plotting]{Python for Science and Engg: Interactive Plotting}

\author[FOSSEE] {FOSSEE}

\institute[IIT Bombay] {Department of Aerospace Engineering\\IIT Bombay}
\date[] {31, October 2009\\Day 1, Session 1}
%%%%%%%%%%%%%%%%%%%%%%%%%%%%%%%%%%%%%%%%%%%%%%%%%%%%%%%%%%%%%%%%%%%%%%

%\pgfdeclareimage[height=0.75cm]{iitmlogo}{iitmlogo}
%\logo{\pgfuseimage{iitmlogo}}


%% Delete this, if you do not want the table of contents to pop up at
%% the beginning of each subsection:
\AtBeginSubsection[]
{
  \begin{frame}<beamer>
    \frametitle{Outline}
    \tableofcontents[currentsection,currentsubsection]
  \end{frame}
}

\AtBeginSection[]
{
  \begin{frame}<beamer>
    \frametitle{Outline}
    \tableofcontents[currentsection,currentsubsection]
  \end{frame}
}

% If you wish to uncover everything in a step-wise fashion, uncomment
% the following command: 
%\beamerdefaultoverlayspecification{<+->}

%\includeonlyframes{current,current1,current2,current3,current4,current5,current6}

%%%%%%%%%%%%%%%%%%%%%%%%%%%%%%%%%%%%%%%%%%%%%%%%%%%%%%%%%%%%%%%%%%%%%%
% DOCUMENT STARTS
\begin{document}

\begin{frame}
  \maketitle
\end{frame}

%% \begin{frame}
%%   \frametitle{Outline}
%%   \tableofcontents
%%   % You might wish to add the option [pausesections]
%% \end{frame}

\begin{frame}
  \frametitle{Workshop Schedule: Day 1}
  \begin{description}
	\item[Session 1] Sat 09:00--10:00
	\item[Session 2] Sat 10:05--11:05
	\item[Session 3] Sat 11:20--12:20
	\item[Session 4] Sat 12:25--13:25
        \item[Quiz -1]   Sat 14:25--14:40
        \item[Session 5] Sat 14:40--15:40
        \item[Session 6] Sat 15:55--16:55
        \item[Quiz -2]   Sat 17:00--17:15
  \end{description}
\end{frame}

\begin{frame}
  \frametitle{Workshop Schedule: Day 2}
  \begin{description}
	\item[Session 1] Sun 09:00--10:00
	\item[Session 2] Sun 10:05--11:05
	\item[Session 3] Sun 11:20--12:20
	\item[Session 4] Sun 12:25--13:25
        \item[Quiz -1]   Sun 14:25--14:40
        \item[Session 5] Sun 14:40--15:40
        \item[Session 6] Sun 15:55--16:55
        \item[Quiz -2]   Sun 17:00--17:15
  \end{description}
\end{frame}

\begin{frame}
  \frametitle{About the Workshop}
  \begin{block}{Intended Audience}
  \begin{itemize}
       \item Engg., Mathematics and Science teachers.
       \item Interested students from similar streams.
  \end{itemize}
  \end{block}  

  \begin{block}{Goal: Successful participants will be able to}
    \begin{itemize}
      \item Use Python as plotting, computational toolkit
      \item Understand how Python can be used as scripting and problem solving language.
      \item Train the students to use Python for the same
    \end{itemize}
  \end{block}
\end{frame}

\section{Getting started}
\begin{frame}
\frametitle{Checklist}
  \begin{enumerate}
    \item IPython: Type ipython at the command line. Is it available?
    \item Editor: We recommend scite.
    \item Data files: Make sure you have all data files.
  \end{enumerate}
\end{frame}

\begin{frame}[fragile]
\frametitle{Starting up...}
\begin{block}{}
\begin{verbatim}
  $ ipython -pylab  
\end{verbatim}
\end{block}
\begin{lstlisting}     
  In []: print "Hello, World!"
  Hello, World!
\end{lstlisting}
Exiting
\begin{lstlisting}     
  In []: ^D(Ctrl-D)
  Do you really want to exit([y]/n)? y
\end{lstlisting}
\end{frame}

\begin{frame}[fragile]
\frametitle{Loops}
Breaking out of loops
\begin{lstlisting}     
  In []: while True:
    ...:     print "Hello, World!"
    ...:     
  Hello, World!
  Hello, World!^C(Ctrl-C)
  ------------------------------------
  KeyboardInterrupt                   

\end{lstlisting}
\end{frame}

\section{Plotting}

\subsection{Drawing plots}
\begin{frame}[fragile]
\frametitle{First Plot}
\begin{columns}
    \column{0.25\textwidth}
    \hspace*{-0.5in}
  \includegraphics[height=2in, interpolate=true]{data/firstplot}
    \column{0.8\textwidth}
    \begin{block}{}
    \begin{small}
\begin{lstlisting}
In []: x = linspace(0, 2*pi, 51)
In []: plot(x, sin(x))
\end{lstlisting}
    \end{small}
    \end{block}
\end{columns}
\end{frame}


\begin{frame}[fragile]
\frametitle{Walkthrough}
\begin{block}{\typ{x = linspace(start, stop, num)} }
returns \typ{num} evenly spaced points, in the interval [\typ{start}, \typ{stop}].
\end{block}
\begin{lstlisting}
x[0] = start
x[num - 1] = end
\end{lstlisting}
\vspace*{.35in}
\begin{block}{\typ{plot(x, y)}}
plots \typ{x} and \typ{y} using default line style and color
\end{block}
\end{frame}

\subsection{Decoration}
\begin{frame}[fragile]
\frametitle{Adding Labels}
\begin{columns}
  \column{0.25\textwidth}
  \hspace*{-0.45in}
  \includegraphics[height=2in, interpolate=true]{data/label}  
  \hspace*{0.5in}
  \column{0.55\textwidth}
  \begin{block}{}
  \small
  \begin{lstlisting}
In []: xlabel('x')

In []: ylabel('sin(x)')
  \end{lstlisting}
  \small
%  \end{lstlisting}
%\typ{xlabel(s)} sets the label of the \typ{x}-axis to \typ{s}

%  \begin{lstlisting}
  \end{block}
%\typ{ylabel(s)} sets the label of the \typ{y}-axis to \typ{s}
\end{columns}
\end{frame}

\begin{frame}[fragile]
\frametitle{Another example}
  \begin{lstlisting}
In []: clf()
#Clears the plot area.
In []: y = linspace(0, 2*pi, 51)
In []: plot(y, sin(2*y))
In []: xlabel('y')
In []: ylabel('sin(2y)')
  \end{lstlisting}
\emphbar{By default plots would be overlaid!}
\end{frame}

\subsection{More decoration}
\begin{frame}[fragile]
\frametitle{Title and Legends}
\vspace*{-0.15in}
%  \begin{block}{}
%  \small
\begin{lstlisting}
In []: title('Sinusoids')
#Sets the title of the figure
In []: legend(['sin(2y)'])
\end{lstlisting}
%  \small
%  \end{block}
  \vspace*{-0.1in}
  \begin{center}
  \includegraphics[height=2in, interpolate=true]{data/legend}  
  \end{center}
\end{frame}

\begin{frame}[fragile]
\frametitle{Legend Placement}

\begin{block}{}
    \small
\begin{lstlisting}
In []: legend(['sin(2y)'], loc='center')
\end{lstlisting}
\end{block}

\begin{columns}
    \column{0.6\textwidth}
 \includegraphics[height=2in, interpolate=true]{data/position}  
\begin{lstlisting}
'best', 'right', 'center'
\end{lstlisting}
\column{0.45\textwidth}
\vspace{-0.2in}
\begin{lstlisting}
'upper right'     
'upper left'      
'lower left'      
'lower right'     
'center left'     
'center right'    
'lower center'    
'upper center'    
\end{lstlisting}
\end{columns}
\end{frame}

\begin{frame}[fragile]
  \frametitle{For arbitrary location}
\vspace*{-0.1in}
\begin{lstlisting}
In []: legend(['sin(2y)'], loc=(.8,.1)) 
# Specify south-east corner position
\end{lstlisting}
%\vspace*{-0.2in}
\begin{center}
  \includegraphics[height=2in, interpolate=true]{data/loc}  
\end{center}
\end{frame}

\begin{frame}[fragile]
\frametitle{Saving \& Closing}
\begin{lstlisting}
In []: savefig('sin.png')

In []: close()
\end{lstlisting}
\end{frame}

\section{Multiple plots}
\begin{frame}[fragile]
\frametitle{Plotting separate figures}
\begin{lstlisting}
In []: figure(1)
In []: plot(y, sin(y))
In []: figure(2)
In []: plot(y, cos(y))
In []: figure(1)
In []: title('sin(y)')
In []: close()
In []: close()
\end{lstlisting}
\end{frame}

\begin{frame}[fragile]
\frametitle{Showing it better}
\vspace{-0.15in}
\begin{lstlisting}
In []: plot(y, sin(y), 'g')

In []: clf()
In []: plot(y, sin(y), linewidth=2)
\end{lstlisting}
\vspace*{-0.2in}
\begin{center}
  \includegraphics[height=2.2in, interpolate=true]{data/green}  
\end{center}
\end{frame}

\begin{frame}[fragile]
\frametitle{Annotating}
\vspace*{-0.15in}
\begin{lstlisting}
In []: annotate('local max', 
       xy=(1.5, 1), 
       xytext=(2.5, .8),
       arrowprops=dict(
       shrink=0.05),)
\end{lstlisting}
\vspace*{-0.2in}
\begin{center}
  \includegraphics[height=2in, interpolate=true]{data/annotate}  
\end{center}
\end{frame}

\begin{frame}[fragile]
\frametitle{Axes lengths}
  \begin{lstlisting}
#Get the axes limits
In []: xmin, xmax = xlim() 
In []: ymin, ymax = ylim() 

In []: xmax = 2*pi
#Set the axes limits
In []: xlim(xmin, xmax) 
In []: ylim(ymin, ymax) 
  \end{lstlisting}
\end{frame}

\begin{frame}[fragile]
\frametitle{Review Problem}
\begin{enumerate}
\item Plot x, -x, sin(x), xsin(x) in range $-5\pi$ to $5\pi$
\item Add a legend
\item Annotate the origin
\item Set axis limits to the range of x
\end{enumerate}
\begin{lstlisting}
In []: x=linspace(-5*pi, 5*pi, 501)
In []: plot(x, x, 'b')
In []: plot(x, -x, 'b')
\end{lstlisting}
$\vdots$
\end{frame}

\section{Exercises}
\begin{frame}[fragile]
\frametitle{Review Problem \ldots}
\small{
\begin{lstlisting}
In []: plot(x, sin(x), 'g', linewidth=2)
In []: plot(x, x*sin(x), 'r', linewidth=3)
\end{lstlisting}

\begin{lstlisting}
In []: legend(['x', '-x', 'sin(x)', 'xsin(x)'])
In []: annotate('origin', 
                 xy=(0, 0), 
                 xytext=(0, -7),
                 arrowprops=dict(
                 shrink=0.05))
In []: xlim(5*pi, 5*pi)
In []: ylim(5*pi, 5*pi)
\end{lstlisting}
}
\end{frame}
\begin{frame}
  \frametitle{What did we learn?}
  \begin{itemize}
    \item Creating simple plots.
    \item Adding labels and legends.
    \item Annotating plots.
    \item Changing the looks: size, linewidth
  \end{itemize}
  \begin{block}{Note}
    \centerline{\alert{Don't Close \typ{IPython}}}
  \end{block}
\end{frame}

\end{document}

