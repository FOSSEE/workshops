%%%%%%%%%%%%%%%%%%%%%%%%%%%%%%%%%%%%%%%%%%%%%%%%%%%%%%%%%%%%%%%%%%%%%%%%%%%%%%%%
% Tutorial slides on Python.
%
% Author: Prabhu Ramachandran <prabhu at aero.iitb.ac.in>
% Copyright (c) 2005-2009, Prabhu Ramachandran
%%%%%%%%%%%%%%%%%%%%%%%%%%%%%%%%%%%%%%%%%%%%%%%%%%%%%%%%%%%%%%%%%%%%%%%%%%%%%%%%

\documentclass[14pt,compress]{beamer}
%\documentclass[draft]{beamer}
%\documentclass[compress,handout]{beamer}
%\usepackage{pgfpages} 
%\pgfpagesuselayout{2 on 1}[a4paper,border shrink=5mm]

% Modified from: generic-ornate-15min-45min.de.tex
\mode<presentation>
{
  \usetheme{Warsaw}
  \useoutertheme{split}
  \setbeamercovered{transparent}
}

\usepackage[english]{babel}
\usepackage[latin1]{inputenc}
%\usepackage{times}
\usepackage[T1]{fontenc}

% Taken from Fernando's slides.
\usepackage{ae,aecompl}
\usepackage{mathpazo,courier,euler}
\usepackage[scaled=.95]{helvet}

\definecolor{darkgreen}{rgb}{0,0.5,0}

\usepackage{listings}
\lstset{language=Python,
    basicstyle=\ttfamily\bfseries,
    commentstyle=\color{red}\itshape,
  stringstyle=\color{darkgreen},
  showstringspaces=false,
  keywordstyle=\color{blue}\bfseries}

%%%%%%%%%%%%%%%%%%%%%%%%%%%%%%%%%%%%%%%%%%%%%%%%%%%%%%%%%%%%%%%%%%%%%%
% Macros
\setbeamercolor{emphbar}{bg=blue!20, fg=black}
\newcommand{\emphbar}[1]
{\begin{beamercolorbox}[rounded=true]{emphbar} 
      {#1}
 \end{beamercolorbox}
}
\newcounter{time}
\setcounter{time}{0}
\newcommand{\inctime}[1]{\addtocounter{time}{#1}{\tiny \thetime\ m}}

\newcommand{\typ}[1]{\lstinline{#1}}

\newcommand{\kwrd}[1]{ \texttt{\textbf{\color{blue}{#1}}}  }

%%% This is from Fernando's setup.
% \usepackage{color}
% \definecolor{orange}{cmyk}{0,0.4,0.8,0.2}
% % Use and configure listings package for nicely formatted code
% \usepackage{listings}
% \lstset{
%    language=Python,
%    basicstyle=\small\ttfamily,
%    commentstyle=\ttfamily\color{blue},
%    stringstyle=\ttfamily\color{orange},
%    showstringspaces=false,
%    breaklines=true,
%    postbreak = \space\dots
% }

%%%%%%%%%%%%%%%%%%%%%%%%%%%%%%%%%%%%%%%%%%%%%%%%%%%%%%%%%%%%%%%%%%%%%%
% Title page
\title[]{Interactive Plotting}

\author[FOSSEE] {FOSSEE}

\institute[IIT Bombay] {Department of Aerospace Engineering\\IIT Bombay}
\date[] {31, October 2009\\Day 1, Session 1}
%%%%%%%%%%%%%%%%%%%%%%%%%%%%%%%%%%%%%%%%%%%%%%%%%%%%%%%%%%%%%%%%%%%%%%

%\pgfdeclareimage[height=0.75cm]{iitmlogo}{iitmlogo}
%\logo{\pgfuseimage{iitmlogo}}


%% Delete this, if you do not want the table of contents to pop up at
%% the beginning of each subsection:
\AtBeginSubsection[]
{
  \begin{frame}<beamer>
    \frametitle{Outline}
    \tableofcontents[currentsection,currentsubsection]
  \end{frame}
}

\AtBeginSection[]
{
  \begin{frame}<beamer>
    \frametitle{Outline}
    \tableofcontents[currentsection,currentsubsection]
  \end{frame}
}

% If you wish to uncover everything in a step-wise fashion, uncomment
% the following command: 
%\beamerdefaultoverlayspecification{<+->}

%\includeonlyframes{current,current1,current2,current3,current4,current5,current6}

%%%%%%%%%%%%%%%%%%%%%%%%%%%%%%%%%%%%%%%%%%%%%%%%%%%%%%%%%%%%%%%%%%%%%%
% DOCUMENT STARTS
\begin{document}

\begin{frame}
  \maketitle
\end{frame}

%% \begin{frame}
%%   \frametitle{Outline}
%%   \tableofcontents
%%   % You might wish to add the option [pausesections]
%% \end{frame}


\begin{frame}[fragile]
\frametitle{Starting up...}
\begin{verbatim}
  $ ipython -pylab  
\end{verbatim}
Exiting
\begin{lstlisting}     
  In []: print "Hello, World!"
  In []: ^D
  Do you really want to exit ([y]/n)? y
\end{lstlisting}
Breaking out of loops
\begin{lstlisting}     
  In []: while True:
    ...:     print "Hello, World!"
    ...:     
  Hello, World!
  Hello, World!^C
\end{lstlisting}
\end{frame}

\begin{frame}[fragile]
\frametitle{First Plot}
\begin{lstlisting}
  In []: x = linspace(0, 2*pi, 51)
\end{lstlisting}
\typ{linspace(start, stop, num)} \\
returns \typ{num} evenly spaced points, in the interval [\typ{start}, \typ{stop}].
\begin{lstlisting}

  In []: plot(x,sin(x))
\end{lstlisting}
\typ{plot(x, y)}\\
plots \typ{x} and \typ{y} using default line style and color
\end{frame}

\begin{frame}[fragile]
\frametitle{Adding Labels}
  \begin{lstlisting}
In []: xlabel('x')
  \end{lstlisting}
\typ{xlabel(s)} sets the label of the \typ{x}-axis to \typ{s}

  \begin{lstlisting}
In []: ylabel('sin(x)')
  \end{lstlisting}
\typ{ylabel(s)} sets the label of the \typ{y}-axis to \typ{s}
\end{frame}

\begin{frame}[fragile]
\frametitle{Another example}
  \begin{lstlisting}
In []: clf()
In []: y = linspace(0, 2*pi, 51)
In []: plot(y, -2*sin(-y))
In []: xlabel('y')
In []: ylabel('-2sin(-y)')
  \end{lstlisting}
\end{frame}

\begin{frame}[fragile]
\frametitle{Title and Legends}
\begin{lstlisting}
In []: title('Sinusoids')
#Sets the title of the figure

In []: legend() 
# When plot made with label
# plot(y, -2*sin(-y), label='sin')
In []: legend(['sin'])
# When no label, or to change

\end{lstlisting}
\end{frame}

\begin{frame}[fragile]
\frametitle{Changing Legend Placement}
\begin{lstlisting}
In []: legend(['sin'], loc=5) 
#or 
In []: legend(['sin'], loc='right') 
#or
In []: legend(['sin'], loc=(x,y)) 
#(x,y) is position of lower-left 
#corner of legend in the axes co-ords
\end{lstlisting}
\end{frame}

\begin{frame}[fragile]
\frametitle{Changing Legend Placement}
\vspace{-0.15in}
\begin{lstlisting}
Location String   Code
===============   ====
'best'            0
'upper right'     1
'upper left'      2
'lower left'      3
'lower right'     4
'right'           5
'center left'     6
'center right'    7
'lower center'    8
'upper center'    9
'center'          10
\end{lstlisting}
\end{frame}


\begin{frame}[fragile]
\frametitle{Saving \& Closing}
\begin{lstlisting}
In []: savefig('sin.png')

In []: close()
\end{lstlisting}
\end{frame}

\begin{frame}[fragile]
\frametitle{Multiple Figures}
\begin{lstlisting}
In []: figure(1)
In []: plot(x, sin(x))
In []: figure(2)
In []: plot(x, cos(x))
In []: figure(1)
In []: title('sin(x)'))
\end{lstlisting}
\end{frame}

\begin{frame}[fragile]
\frametitle{Showing it better}
\begin{lstlisting}
In []: plot(y, sin(y), 'g')
# plots the curve using green color

In []: plot(y, sin(y), linewidth=2)
# sets the linewidth to 2  
\end{lstlisting}
\end{frame}

\begin{frame}[fragile]
\frametitle{Annotating}
\begin{lstlisting}
In []: annotate('Sample point',
                (50,200))
# Adds the note 'Sample point' at 
# the point (50, 200)
\end{lstlisting}
\end{frame}

\begin{frame}[fragile]
\frametitle{Axes lengths}
  \begin{lstlisting}
#Get the axes limits
In []: xmin, xmax = xlim() 
In []: ymin, ymax = ylim() 

#Set the axes limits
In []: xlim( xmin, xmax ) 
In []: ylim( ymin, ymax ) 
  \end{lstlisting}
\end{frame}

\begin{frame}[fragile]
  \begin{center}
  End of Session-1\\
  \alert{Don't Close \typ{IPython}}
  \end{center}
\end{frame}

\end{document}
