%%%%%%%%%%%%%%%%%%%%%%%%%%%%%%%%%%%%%%%%%%%%%%%%%%%%%%%%%%%%%%%%%%%%%%%%%%%%%%%%
% Tutorial slides on Python.
%
% Author: FOSSEE <info at fossee  dot in>
% Copyright (c) 2005-2009, FOSSEE Team
%%%%%%%%%%%%%%%%%%%%%%%%%%%%%%%%%%%%%%%%%%%%%%%%%%%%%%%%%%%%%%%%%%%%%%%%%%%%%%%%


\documentclass[14pt,compress]{beamer}

\mode<presentation>
{
  \useoutertheme{split}
  \setbeamercovered{transparent}
}

\definecolor{darkgreen}{rgb}{0,0.5,0}

\usepackage{listings}
\lstset{language=Python,
    basicstyle=\ttfamily\bfseries,
    commentstyle=\color{red}\itshape,
  stringstyle=\color{darkgreen},
  showstringspaces=false,
  keywordstyle=\color{blue}\bfseries}

\newcommand{\kwrd}[1]{ \texttt{\textbf{\color{blue}{#1}}}  }

%%%%%%%%%%%%%%%%%%%%%%%%%%%%%%%%%%%%%%%%%%%%%%%%%%%%%%%%%%%%%%%%%%%%%%
% Macros

\newcounter{qno}
\setcounter{qno}{0}
\newcommand{\incqno}{\addtocounter{qno}{1}{Question \theqno}}

%%%%%%%%%%%%%%%%%%%%%%%%%%%%%%%%%%%%%%%%%%%%%%%%%%%%%%%%%%%%%%%%%%%%%%
% Title page
\title[Basic Python]{Python for science and engineering: Day 1, Quiz 2}

\author[FOSSEE Team] {FOSSEE}

\institute[IIT Bombay] {Department of Aerospace Engineering\\IIT Bombay}
\date[] {\today \\
Day 1, quiz 2}
%%%%%%%%%%%%%%%%%%%%%%%%%%%%%%%%%%%%%%%%%%%%%%%%%%%%%%%%%%%%%%%%%%%%%%


\begin{document}

\begin{frame}
  \titlepage
\end{frame}

\begin{frame}
  \frametitle{Write your details...}
On the top right hand corner please write down the following:
  \begin{itemize}
    \item Name:
    \item University/College/Company:
    \item Student/Teacher/Professional:
  \end{itemize}
\end{frame}

\begin{frame}[fragile]
\frametitle{\incqno }
\begin{lstlisting}
In []: a = array([[1, 2],
                  [3, 4]])
In []: a[1,0] = 0
\end{lstlisting}
What is the resulting array?
\end{frame}

\begin{frame}[fragile]
\frametitle{\incqno }
\begin{lstlisting}
  In []: x = array(([1,2,3,4],
                    [2,3,4,5]))
  In []: x[-2][-3] = 4
  In []: print x
\end{lstlisting}
What will be printed?
\end{frame}

%% \begin{frame}[fragile]
%% \frametitle{\incqno }
%% \begin{lstlisting}
%%   In []: x = array([[1,2,3,4],
%%                     [3,4,2,5]])
%% \end{lstlisting}
%% What is the \lstinline+shape+ of this array?
%% \end{frame}

\begin{frame}[fragile]
\frametitle{\incqno }
\begin{lstlisting}
  In []: x = array([[1,2,3,4]])
\end{lstlisting}
How to change \lstinline+x+ to \lstinline+array([[1,2,0,4]])+?
\end{frame}

\begin{frame}[fragile]
\frametitle{\incqno }
\begin{lstlisting}
  In []: x = array([[1,2,3,4],
                    [3,4,2,5]])
\end{lstlisting}
How do you get the following slice of \lstinline+x+?
\begin{lstlisting}
array([[2,3],
       [4,2]])
\end{lstlisting}
\end{frame}

\begin{frame}[fragile]
\frametitle{\incqno }
\begin{lstlisting}
  In []: x = array([[9,18,27],
                    [30,60,90],
                    [14,7,1]])
\end{lstlisting}
What is the output of \lstinline+x[::3,::3]+
\end{frame}


\begin{frame}[fragile]
\frametitle{\incqno }
\begin{lstlisting}
In []: a = array([[1, 2],
                  [3, 4]])
\end{lstlisting}
How do you get the transpose of this array?
\end{frame}

\begin{frame}[fragile]
\frametitle{\incqno }
\begin{lstlisting}
In []: a = array([[1, 2],
                  [3, 4]])
In []: b = array([[1, 1],
                  [2, 2]])
In []: a*b
\end{lstlisting}
What does this produce?
\end{frame}

\begin{frame}
\frametitle{\incqno }
What command do you use to find the inverse of a matrix and its
eigenvalues?
\end{frame}

%% \begin{frame}
%% \frametitle{\incqno }
%% The file \lstinline+datafile.txt+ contains 3 columns of data.  What
%% command will you use to read the entire data file into an array?
%% \end{frame}

%% \begin{frame}
%% \frametitle{\incqno }
%% If the contents of the file \lstinline+datafile.txt+ is read into an
%% $N\times3$ array called \lstinline+data+, how would you obtain the third
%% column of this data?
%% \end{frame}

\begin{frame}
\frametitle{\incqno }
Given a 4x4 matrix \texttt{A} and a 4-vector \texttt{b}, what command do
you use to solve for the equation \\
\texttt{Ax = b}?
\end{frame}

\begin{frame}
\frametitle{\incqno }
What command will you use if you wish to integrate a system of ODEs?
\end{frame}

\begin{frame}
\frametitle{\incqno }
How do you calculate the roots of the polynomial, $y = 1 + 6x + 8x^2 +
x^3$?
\end{frame}

\begin{frame}
\frametitle{\incqno }
Two arrays \lstinline+a+ and \lstinline+b+ are numerically almost equal, what command
do you use to check if this is true?
\end{frame}

%% \begin{frame}[fragile]
%% \frametitle{\incqno }
%% \begin{lstlisting}
%%   In []: x = arange(0, 1, 0.25)
%%   In []: print x
%% \end{lstlisting}
%% What will be printed?
%% \end{frame}


%% \begin{frame}[fragile]
%% \frametitle{\incqno }
%% \begin{lstlisting}
%% from scipy.integrate import quad
%% def f(x):
%%     res = x*cos(x)
%% quad(f, 0, 1)
%% \end{lstlisting}
%% What changes will you make to the above code to make it work?
%% \end{frame}

%% \begin{frame}
%% \frametitle{\incqno }
%% What two commands will you use to create and evaluate a spline given
%% some data?
%% \end{frame}

%% \begin{frame}[fragile]
%% \frametitle{\incqno }
%% What would be the result?
%% \begin{lstlisting}
%%   In []: x
%%   array([[0, 1, 2],
%%          [3, 4, 5],
%%          [6, 7, 8]])
%%   In []: x[::-1,:]
%% \end{lstlisting}
%% Hint:
%% \begin{lstlisting}
%%   In []: x = arange(9)
%%   In []: x[::-1]
%%   array([8, 7, 6, 5, 4, 3, 2, 1, 0])
%% \end{lstlisting}
%% \end{frame}

%% \begin{frame}[fragile]
%% \frametitle{\incqno }
%% What would be the result?
%% \begin{lstlisting}
%%   In []: y = arange(3)
%%   In []: x = linspace(0,3,3)
%%   In []: x-y
%% \end{lstlisting}
%% \end{frame}

%% \begin{frame}[fragile]
%% \frametitle{\incqno }
%% \begin{lstlisting}
%%   In []: x
%%   array([[ 0, 1, 2, 3],
%%          [ 4, 5, 6, 7],
%%          [ 8, 9, 10, 11],
%%          [12, 13, 14, 15]])
%% \end{lstlisting}
%% How will you get the following \lstinline+x+?
%% \begin{lstlisting}
%%   array([[ 5, 7],
%%          [ 9, 11]])
%% \end{lstlisting}
%% \end{frame}

%% \begin{frame}[fragile]
%% \frametitle{\incqno }
%% What would be the output?
%% \begin{lstlisting}
%%   In []: y = arange(4)
%%   In []: x = array(([1,2,3,2],[1,3,6,0]))
%%   In []: x + y
%% \end{lstlisting}
%% \end{frame}

%% \begin{frame}[fragile]
%% \frametitle{\incqno }
%% \begin{lstlisting}
%%   In []: line = plot(x, sin(x))
%% \end{lstlisting}
%% Use the \lstinline+set_linewidth+ method to set width of \lstinline+line+ to 2.
%% \end{frame}

%% \begin{frame}[fragile]
%% \frametitle{\incqno }
%% What would be the output?
%% \begin{lstlisting}
%%   In []: x = arange(9)
%%   In []: y = arange(9.)
%%   In []: x == y
%% \end{lstlisting}
%% \end{frame}


\end{document}


