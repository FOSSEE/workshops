%%%%%%%%%%%%%%%%%%%%%%%%%%%%%%%%%%%%%%%%%%%%%%%%%%%%%%%%%%%%%%%%%%%%%%%%%%%%%%%%
%Tutorial slides on Python.
%
% Author: Prabhu Ramachandran <prabhu at aero.iitb.ac.in>
% Copyright (c) 2005-2009, Prabhu Ramachandran
%%%%%%%%%%%%%%%%%%%%%%%%%%%%%%%%%%%%%%%%%%%%%%%%%%%%%%%%%%%%%%%%%%%%%%%%%%%%%%%%

\documentclass[14pt,compress]{beamer}
%\documentclass[draft]{beamer}
%\documentclass[compress,handout]{beamer}
%\usepackage{pgfpages} 
%\pgfpagesuselayout{2 on 1}[a4paper,border shrink=5mm]

% Modified from: generic-ornate-15min-45min.de.tex
\mode<presentation>
{
  \usetheme{Warsaw}
  \useoutertheme{infolines}
  \setbeamercovered{transparent}
}

\usepackage[english]{babel}
\usepackage[latin1]{inputenc}
%\usepackage{times}
\usepackage[T1]{fontenc}

% Taken from Fernando's slides.
\usepackage{ae,aecompl}
\usepackage{mathpazo,courier,euler}
\usepackage[scaled=.95]{helvet}

\definecolor{darkgreen}{rgb}{0,0.5,0}

\usepackage{listings}
\lstset{language=Python,
    basicstyle=\ttfamily\bfseries,
    commentstyle=\color{red}\itshape,
  stringstyle=\color{darkgreen},
  showstringspaces=false,
  keywordstyle=\color{blue}\bfseries}

%%%%%%%%%%%%%%%%%%%%%%%%%%%%%%%%%%%%%%%%%%%%%%%%%%%%%%%%%%%%%%%%%%%%%%
% Macros
\setbeamercolor{emphbar}{bg=blue!20, fg=black}
\newcommand{\emphbar}[1]
{\begin{beamercolorbox}[rounded=true]{emphbar} 
      {#1}
 \end{beamercolorbox}
}
\newcounter{time}
\setcounter{time}{0}
\newcommand{\inctime}[1]{\addtocounter{time}{#1}{\tiny \thetime\ m}}

\newcommand{\typ}[1]{\texttt{#1}}

\newcommand{\kwrd}[1]{ \texttt{\textbf{\color{blue}{#1}}}  }

%%% This is from Fernando's setup.
% \usepackage{color}
% \definecolor{orange}{cmyk}{0,0.4,0.8,0.2}
% % Use and configure listings package for nicely formatted code
% \usepackage{listings}
% \lstset{
%    language=Python,
%    basicstyle=\small\ttfamily,
%    commentstyle=\ttfamily\color{blue},
%    stringstyle=\ttfamily\color{orange},
%    showstringspaces=false,
%    breaklines=true,
%    postbreak = \space\dots
% }


%%%%%%%%%%%%%%%%%%%%%%%%%%%%%%%%%%%%%%%%%%%%%%%%%%%%%%%%%%%%%%%%%%%%%%
% Title page
\title[Basic Python]{Python language: Basics}

\author[FOSSEE Team] {The FOSSEE Group}

\institute[IIT Bombay] {Department of Aerospace Engineering\\IIT Bombay}
\date[] {1, November 2009\\Day 2, Session 1}
%%%%%%%%%%%%%%%%%%%%%%%%%%%%%%%%%%%%%%%%%%%%%%%%%%%%%%%%%%%%%%%%%%%%%%

%\pgfdeclareimage[height=0.75cm]{iitmlogo}{iitmlogo}
%\logo{\pgfuseimage{iitmlogo}}


%% Delete this, if you do not want the table of contents to pop up at
%% the beginning of each subsection:
\AtBeginSubsection[]
{
  \begin{frame}<beamer>
    \frametitle{Outline}
    \tableofcontents[currentsection,currentsubsection]
  \end{frame}
}

\AtBeginSection[]
{
  \begin{frame}<beamer>
    \frametitle{Outline}
    \tableofcontents[currentsection,currentsubsection]
  \end{frame}
}

% If you wish to uncover everything in a step-wise fashion, uncomment
% the following command: 
%\beamerdefaultoverlayspecification{<+->}

%\includeonlyframes{current,current1,current2,current3,current4,current5,current6}

%%%%%%%%%%%%%%%%%%%%%%%%%%%%%%%%%%%%%%%%%%%%%%%%%%%%%%%%%%%%%%%%%%%%%%
% DOCUMENT STARTS
\begin{document}

\begin{frame}
  \titlepage
\end{frame}

\begin{frame}
  \frametitle{Outline}
  \tableofcontents
  % You might wish to add the option [pausesections]
\end{frame}

\section{Data types}

\begin{frame}
  \frametitle{Primitive Data types}
  \begin{itemize}
    \item Numbers: float, int, complex
    \item Strings
    \item Boolean
  \end{itemize}
\end{frame}

\subsection{Numbers}
\begin{frame}[fragile]
  \frametitle{Numbers}
  \begin{itemize}
    \item \kwrd{int}\\ \kwrd{int} = whole number, no matter what the size!
  \begin{lstlisting}
In [1]: a = 13

In [2]: b = 99999999999999999999
  \end{lstlisting}
    \item \kwrd{float}
  \begin{lstlisting}
In [3]: fl = 3.141592
  \end{lstlisting}
  \end{itemize}
\end{frame}

\begin{frame}[fragile]
\frametitle{Complex numbers}
  \begin{lstlisting}
In [1]: cplx = 3+4j

In [2]: abs(cplx)
Out[2]: 5.0

In [3]: cplx.imag
Out[3]: 4.0

In [4]: cplx.real
Out[4]: 3.0
  \end{lstlisting}
\end{frame}

\subsection{Boolean}
\begin{frame}[fragile]
  \frametitle{Boolean}
  \begin{lstlisting}
In [1]: t = True

In [2]: f = not t
Out[2]: False

In [3]: f or t
Out[3]: True

In [4]: f and t
Out[4]: False
  \end{lstlisting}
  \inctime{5}
\end{frame}

\subsection{Strings}

\begin{frame}[fragile]
  \frametitle{Strings}
Strings were introduced previously, let us now look at them in a little more detail.
  \begin{lstlisting}
In [1]: w = "hello"

In [2]: print w[0] + w[2] + w[-1]
Out[2]: hlo

In [3]: len(w) # guess what
Out[3]: 5
  \end{lstlisting}
\end{frame}

\begin{frame}[fragile]
  \frametitle{Strings \ldots}
  \begin{lstlisting}
In [1]: w[0] = 'H' # Can't do that!
--------------------------------------------
TypeError  Traceback (most recent call last)

/<ipython console> in <module>()

TypeError: 'str' object does not
         support item assignment
  \end{lstlisting}
\end{frame}

\begin{frame}[fragile]
  \frametitle{String methods}
  \begin{lstlisting}
In [1]: a = 'Hello World'
In [2]: a.startswith('Hell')
Out[2]: True

In [3]: a.endswith('ld')
Out[3]: True

In [4]: a.upper()
Out[4]: 'HELLO WORLD'

In [5]: a.lower()
Out[5]: 'hello world'
  \end{lstlisting}
\end{frame}

\begin{frame}[fragile]
\frametitle{Still with strings}
  \begin{itemize}
    \item We saw split() yesterday
    \item join() is the opposite of split()
  \end{itemize}
  \begin{lstlisting}
In [1]: ''.join(['a', 'b', 'c'])
Out[1]: 'abc'
  \end{lstlisting}
\end{frame}

\begin{frame}[fragile]
\frametitle{String formatting}
  \begin{lstlisting}
In [1]: x, y = 1, 1.234

In [2]: 'x is %s, y is %s' %(x, y)
Out[2]: 'x is 1, y is 1.234'
  \end{lstlisting}
  \emphbar{\url{http://docs.python.org/library/stdtypes.html}}
  \inctime{10}
\end{frame}

\section{Operators}
\begin{frame}[fragile]
  \frametitle{Arithematic operators}
  \begin{lstlisting}
In [1]: 1786 % 12
Out[1]: 10

In [2]: 3124 * 126789
Out[2]: 396088836

In [3]: a = 3124 * 126789

In [4]: big = 1234567891234567890 ** 3

In [5]: verybig = big * big * big * big
  \end{lstlisting}
\end{frame}

\begin{frame}[fragile]
  \frametitle{Arithematic operators \ldots}
  \begin{lstlisting}
In [1]: 17/2
Out[1]: 8

In [2]: 17/2.0
Out[2]: 8.5

In [3]: 17.0/2
Out[3]: 8.5

In [4]: 17.0/8.5
Out[4]: 2.0
  \end{lstlisting}
\end{frame}

\begin{frame}[fragile]
  \frametitle{String operations}
  \begin{lstlisting}
In [1]: s = 'Hello '

In [2]: p = 'World'

In [3]: s + p 
Out[3]: 'Hello World'

In [4]: s * 4
Out[4]: 'Hello Hello Hello Hello'
  \end{lstlisting}
\end{frame}

\begin{frame}[fragile]
  \frametitle{String operations \ldots}
  \begin{lstlisting}
In [1]: s * s
--------------------------------------------
TypeError  Traceback (most recent call last)

/<ipython console> in <module>()

TypeError: can't multiply sequence by
                non-int of type 'str'
  \end{lstlisting}
\end{frame}

\begin{frame}[fragile]
  \frametitle{Relational and logical operators}
  \begin{lstlisting}
In [1]: pos, zer, neg = 1, 0, -1
In [2]: pos == neg
Out[2]: False

In [3]: pos >= neg
Out[3]: True

In [4]: neg < zer < pos
Out[4]: True

In [5]: pos + neg != zer
Out[5]: False
  \end{lstlisting}
\end{frame}

\begin{frame}[fragile]
  \frametitle{Built-ins}
  \begin{lstlisting}
In [1]: int(17/2.0)
Out[1]: 8

In [2]: float(17/2)  # Recall
Out[2]: 8.0

In [3]: str(17/2.0)
Out[3]: '8.5'

In [4]: round( 7.5 )
Out[4]: 8.0
  \end{lstlisting}
\end{frame}

\begin{frame}[fragile]
  \frametitle{Odds and ends}
  \begin{itemize}
    \item Case sensitive
    \item Dynamically typed $\Rightarrow$ need not specify a type
      \begin{lstlisting}
In [1]: a = 1
In [2]: a = 1.1
In [3]: a = "Now I am a string!"
      \end{lstlisting}
    \item Comments:
      \begin{lstlisting}
In [4]: a = 1  # In-line comments
In [5]: # A comment line.
In [6]: a = "# Not a comment!"
      \end{lstlisting}
  \end{itemize}
  \inctime{15}
\end{frame}

\section{Simple IO}
\begin{frame}[fragile]
  \frametitle{Simple IO: Console Input}
  \begin{itemize}
    \item raw\_input() waits for user input.
      \begin{lstlisting}
In [1]: a = raw_input()
5

In [2]: a = raw_input('prompt > ')
prompt > 5
      \end{lstlisting}
    \item Prompt string is optional.
    \item All keystrokes are Strings!
    \item \texttt{int()} converts string to int.
  \end{itemize}
\end{frame}

\begin{frame}{Simple IO: Console output}
  \begin{itemize}
    \item \texttt{print} is straight forward
    \item Note the distinction between \texttt{print x} and \texttt{print x,}
  \end{itemize}
\end{frame}

\section{Control flow}
\begin{frame}
  \frametitle{Control flow constructs}  
  \begin{itemize}
  \item \kwrd{if/elif/else}: branching
  \item \kwrd{while}: looping
  \item \kwrd{for}: iterating 
  \item \kwrd{break, continue}: modify loop 
  \item \kwrd{pass}: syntactic filler
  \end{itemize}
\end{frame}

\subsection{Basic Conditional flow}
\begin{frame}[fragile]
  \frametitle{\typ{If...elif...else} example}
  \begin{lstlisting}
x = int(raw_input("Enter an integer:"))
if x < 0:
     print 'Be positive!'
elif x == 0:
     print 'Zero'
elif x == 1:
     print 'Single'
else:
     print 'More'
  \end{lstlisting}
  \inctime{10}
\end{frame}

\subsection{Basic Looping}
\begin{frame}[fragile]
  \frametitle{\typ{while}}
Example: Fibonacci series
  \begin{lstlisting}
# the sum of two elements
# defines the next
a, b = 0, 1
while b < 10:
    print b,
    a, b = b, a + b 
\end{lstlisting}
\typ{1 1 2 3 5 8}\\  
\end{frame}

\begin{frame}[fragile]
\frametitle{\typ{range()}}
\kwrd{range([start,] stop[, step])}\\
\begin{itemize}
  \item range() returns a list of integers
  \item The \emph{start} and the \emph{step} arguments are optional
  \item \emph{stop} argument is not included in the list
\end{itemize}
\end{frame}

\begin{frame}[fragile]
  \frametitle{\typ{for} \ldots \typ{range()}}
Example: print squares of first \typ{n} numbers
  \begin{lstlisting}
In []: for i in range(5):
 ....:     print i, i * i
 ....:
 ....:
0 0
1 1
2 4
3 9
4 16
\end{lstlisting}
\inctime{5}
\end{frame}

\subsection{Exercises}

\begin{frame}{Problem set 1: Problem 1.1}
  Write a program that displays all three digit numbers that are equal to the sum of the cubes of their digits. That is, print numbers $abc$ that have the property $abc = a^3 + b^3 + c^3$\\
\vspace*{0.2in}
\emphbar{These are called $Armstrong$ numbers.}
\end{frame}

\begin{frame}{Problem 1.2 - Collatz sequence}
\begin{enumerate}
  \item Start with an arbitrary (positive) integer. 
  \item If the number is even, divide by 2; if the number is odd, multiply by 3 and add 1.
  \item Repeat the procedure with the new number.
  \item It appears that for all starting values there is a cycle of 4, 2, 1 at which the procedure loops.
\end{enumerate}
    Write a program that accepts the starting value and prints out the Collatz sequence.
\end{frame}

\begin{frame}[fragile]{Problem 1.3}
  Write a program that prints the following pyramid on the screen. 
  \begin{lstlisting}
1
2  2
3  3  3
4  4  4  4
  \end{lstlisting}
The number of lines must be obtained from the user as input.\\
\pause
\emphbar{When can your code fail?}
\inctime{5}
\end{frame}

\begin{frame}[fragile]
  \frametitle{What did we learn?}
  \begin{itemize}
    \item Basic data types
    \item Operators
    \item Conditional structures
    \item Loops
  \end{itemize}
\end{frame}

\end{document}
