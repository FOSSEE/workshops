%%%%%%%%%%%%%%%%%%%%%%%%%%%%%%%%%%%%%%%%%%%%%%%%%%%%%%%%%%%%%%%%%%%%%%%%%%%%%%%%
%Tutorial slides on Python.
%
% Author: FOSSEE 
% Copyright (c) 2009, FOSSEE, IIT Bombay
%%%%%%%%%%%%%%%%%%%%%%%%%%%%%%%%%%%%%%%%%%%%%%%%%%%%%%%%%%%%%%%%%%%%%%%%%%%%%%%%

\documentclass[14pt,compress]{beamer}
%\documentclass[draft]{beamer}
%\documentclass[compress,handout]{beamer}
%\usepackage{pgfpages} 
%\pgfpagesuselayout{2 on 1}[a4paper,border shrink=5mm]

% Modified from: generic-ornate-15min-45min.de.tex
\mode<presentation>
{
  \usetheme{Warsaw}
  \useoutertheme{infolines}
  \setbeamercovered{transparent}
}

\usepackage[english]{babel}
\usepackage[latin1]{inputenc}
%\usepackage{times}
\usepackage[T1]{fontenc}

% Taken from Fernando's slides.
\usepackage{ae,aecompl}
\usepackage{mathpazo,courier,euler}
\usepackage[scaled=.95]{helvet}

\definecolor{darkgreen}{rgb}{0,0.5,0}

\usepackage{listings}
\lstset{language=Python,
    basicstyle=\ttfamily\bfseries,
    commentstyle=\color{red}\itshape,
  stringstyle=\color{darkgreen},
  showstringspaces=false,
  keywordstyle=\color{blue}\bfseries}

%%%%%%%%%%%%%%%%%%%%%%%%%%%%%%%%%%%%%%%%%%%%%%%%%%%%%%%%%%%%%%%%%%%%%%
% Macros
\setbeamercolor{emphbar}{bg=blue!20, fg=black}
\newcommand{\emphbar}[1]
{\begin{beamercolorbox}[rounded=true]{emphbar} 
      {#1}
 \end{beamercolorbox}
}
\newcounter{time}
\setcounter{time}{0}
\newcommand{\inctime}[1]{\addtocounter{time}{#1}{\tiny \thetime\ m}}

\newcommand{\typ}[1]{\lstinline{#1}}

\newcommand{\kwrd}[1]{ \texttt{\textbf{\color{blue}{#1}}}  }

%%% This is from Fernando's setup.
% \usepackage{color}
% \definecolor{orange}{cmyk}{0,0.4,0.8,0.2}
% % Use and configure listings package for nicely formatted code
% \usepackage{listings}
% \lstset{
%    language=Python,
%    basicstyle=\small\ttfamily,
%    commentstyle=\ttfamily\color{blue},
%    stringstyle=\ttfamily\color{orange},
%    showstringspaces=false,
%    breaklines=true,
%    postbreak = \space\dots
% }


%%%%%%%%%%%%%%%%%%%%%%%%%%%%%%%%%%%%%%%%%%%%%%%%%%%%%%%%%%%%%%%%%%%%%%
% Title page
\title[Basic Python]{Python language: Data structures and functions}

\author[FOSSEE Team] {The FOSSEE Group}

\institute[IIT Bombay] {Department of Aerospace Engineering\\IIT Bombay}
\date[] {1 May, 2010\\Day 2, Session 2}
%%%%%%%%%%%%%%%%%%%%%%%%%%%%%%%%%%%%%%%%%%%%%%%%%%%%%%%%%%%%%%%%%%%%%%

%\pgfdeclareimage[height=0.75cm]{iitmlogo}{iitmlogo}
%\logo{\pgfuseimage{iitmlogo}}


%% Delete this, if you do not want the table of contents to pop up at
%% the beginning of each subsection:
\AtBeginSubsection[]
{
  \begin{frame}<beamer>
    \frametitle{Outline}
    \tableofcontents[currentsection,currentsubsection]
  \end{frame}
}

\AtBeginSection[]
{
  \begin{frame}<beamer>
    \frametitle{Outline}
    \tableofcontents[currentsection,currentsubsection]
  \end{frame}
}

% If you wish to uncover everything in a step-wise fashion, uncomment
% the following command: 
%\beamerdefaultoverlayspecification{<+->}

%\includeonlyframes{current,current1,current2,current3,current4,current5,current6}

%%%%%%%%%%%%%%%%%%%%%%%%%%%%%%%%%%%%%%%%%%%%%%%%%%%%%%%%%%%%%%%%%%%%%%
% DOCUMENT STARTS
\begin{document}

\begin{frame}
  \titlepage
\end{frame}

\begin{frame}
  \frametitle{Outline}
  \tableofcontents
  % You might wish to add the option [pausesections]
\end{frame}

\section{Control flow}
\subsection{Basic Looping}
\begin{frame}[fragile]
  \frametitle{\typ{while}}
\begin{block}{Example: Fibonacci series}
  Sum of previous two elements defines the next
\end{block}
  \begin{lstlisting}
In []: a, b = 0, 1
In []: while b < 10:
  ...:     print b,
  ...:     a, b = b, a + b
  ...:
  ...:
\end{lstlisting}
\typ{1 1 2 3 5 8}\\
\end{frame}

\begin{frame}[fragile]
\frametitle{\typ{range()}}
\kwrd{range([start,] stop[, step])}\\
\begin{itemize}
  \item \typ{range()} returns a list of integers
  \item The \typ{start} and the \typ{step} arguments are optional
  \item \typ{stop} is not included in the list
\end{itemize}
\vspace*{.5in}
\begin{block}{Documentation convention}
  \begin{itemize}
    \item \alert{Anything within \typ{[]} is optional}
    \item Nothing to do with Python.
  \end{itemize}
\end{block}
\end{frame}

\begin{frame}[fragile]
  \frametitle{\texttt{for} \ldots \typ{range()}}
Example: print squares of first \typ{5} numbers
  \begin{lstlisting}
In []: for i in range(5):
 ....:     print i, i * i
 ....:
 ....:
0 0
1 1
2 4
3 9
4 16
\end{lstlisting}
\end{frame}

\begin{frame}[fragile]
  \frametitle{\texttt{for} \ldots \typ{range()}}
Example: print squares of odd numbers from 3 to 9
  \begin{lstlisting}
In []: for i in range(3, 10, 2):
 ....:     print i, i * i
 ....:
 ....:
3 9
5 25
7 49
9 81
\end{lstlisting}
\inctime{5}
\end{frame}

\subsection{Exercises}

\begin{frame}{Problem set 1: Problem 1.1}
  Write a program that displays all three digit numbers that are equal to the sum of the cubes of their digits. That is, print numbers $abc$ that have the property $abc = a^3 + b^3 + c^3$\\
For example, $153 = 1^3 + 5^3 + 3^3$\\
\vspace*{0.2in}
\emphbar{These are called $Armstrong$ numbers.}
\end{frame}

\begin{frame}{Problem 1.2 - Collatz sequence}
\begin{enumerate}
  \item Start with an arbitrary (positive) integer. 
  \item If the number is even, divide by 2; if the number is odd, multiply by 3 and add 1.
  \item Repeat the procedure with the new number.
  \item It appears that for all starting values there is a cycle of 4, 2, 1 at which the procedure loops.
\end{enumerate}
    Write a program that accepts the starting value and prints out the Collatz sequence.
\inctime{5}
\end{frame}

\section{Data structures}
\subsection{Lists}
\begin{frame}[fragile]
  \frametitle{Lists}
\begin{block}{We already know that}
  \begin{lstlisting}
num = [1, 2, 3, 4]
  \end{lstlisting}
is a list
\end{block}
\end{frame}

\begin{frame}[fragile]
  \frametitle{Lists: methods}
  \begin{lstlisting}
In []: num = [9, 8, 2, 3, 7]

In []: num + [4, 5, 6]
Out[]: [9, 8, 2, 3, 7, 4, 5, 6]

In []: num.append([4, 5, 6])

In []: num
Out[]: [9, 8, 2, 3, 7, [4, 5, 6]]
  \end{lstlisting}
\end{frame}

\begin{frame}[fragile]
  \frametitle{Lists: methods}
  \begin{lstlisting}
In []: num = [9, 8, 2, 3, 7]

In []: num.extend([4, 5, 6])
In []: num
Out[]: [9, 8, 2, 3, 7, 4, 5, 6]

In []: num.reverse()
In []: num
Out[]: [6, 5, 4, 7, 3, 2, 8, 9]

In []: num.remove(6)
In []: num
  \end{lstlisting}
\end{frame}

\begin{frame}[fragile]
  \frametitle{Lists: slicing}
  \begin{itemize}
    \item \typ{list[initial:final]}
  \end{itemize}
\begin{lstlisting}
In []: a = [1, 2, 3, 4, 5]

In []: a[1:3]
Out[]: [2, 3]

In []: a[1:-1]
Out[]: [2, 3, 4]

In []: a[:3]
Out[]: [1, 2, 3]
\end{lstlisting}
\end{frame}

\begin{frame}[fragile]
  \frametitle{Lists: slicing}
  \begin{itemize}
    \item \typ{list[initial:final:step]}
  \end{itemize}
\begin{lstlisting}
In []: a[1:-1:2]
Out[]: [2, 4]

In []: a[::2]
Out[]: [1, 3, 5]

In []: a[-1::-1]
Out[]: [5, 4, 3, 2, 1]
\end{lstlisting}
\end{frame}

\begin{frame}[fragile]
\frametitle{List containership}
\emphbar{Recall \typ{num} is \typ{[9, 8, 2, 3, 7]}}
\begin{lstlisting}
In []: 4 in num
Out[]: False

In []: b = 8
In []: b in num
Out[]: True

In []: b not in num
Out[]: False
\end{lstlisting}
\end{frame}

\subsection{Tuples}
\begin{frame}[fragile]
\frametitle{Tuples: Immutable lists}
\begin{lstlisting}
In []: t = (1, 2, 3, 4, 5, 6, 7, 8)

In []: t[0] + t[3] + t[-1]
Out[]: 13

In []: t[4] = 7 
\end{lstlisting}
\pause
\begin{block}{Note:}
\begin{itemize}
  \item Tuples are immutable - cannot be changed
\end{itemize}
\end{block}
  \inctime{10}
\end{frame}

\begin{frame}
  {A classic problem}
  \begin{block}
    {Interchange values}
    How to interchange values of two variables? 
  \end{block}
  \pause
  \begin{block}{Note:}
    This Python idiom works for all types of variables.\\
They need not be of the same type!
  \end{block}
\end{frame}

\subsection{Dictionaries}
\begin{frame}[fragile]
  \frametitle{Dictionaries: recall}
  \begin{lstlisting}
In []: player = {'Mat': 134,'Inn': 233,
          'Runs': 10823, 'Avg': 52.53}

In []: player['Avg']
Out[]: 52.530000000000001
  \end{lstlisting}
  \begin{block}{Note!}
    Duplicate keys $\Rightarrow$ overwritten!\\
    You can iterate through a dictionary using keys.
  \end{block}
\end{frame}

\begin{frame}[fragile]
  \frametitle{Dictionaries: containership}
  \begin{lstlisting}
In []: 'Inn' in player
Out[]: True

In []: 'Econ' in player
Out[]: False
  \end{lstlisting}
  \begin{block}{Note}
    \begin{itemize}
      \item We can check for the containership of keys only
      \item Not values
    \end{itemize}
  \end{block}
\end{frame}

\begin{frame}[fragile]
  \frametitle{Dictionaries: methods}
  \begin{lstlisting}
In []: player.keys()
Out[]: ['Runs', 'Inn', 'Avg', 'Mat']

In []: player.values()
Out[]: [10823, 233, 
        52.530000000000001, 134]
  \end{lstlisting}
\end{frame}

\begin{frame} {Problem Set 2.1: Problem 2.1.1}
You are given date strings of the form ``29 Jul, 2009'', or ``4 January 2008''. In other words a number, a string and another number, with a comma sometimes separating the items.\\Write a program that takes such a string as input and prints a tuple (yyyy, mm, dd) where all three elements are ints.
\end{frame}

\subsection{Sets}
\begin{frame}[fragile]
  \frametitle{Sets}
    \begin{itemize}
      \item Simplest container, mutable
      \item No ordering, no duplicates
      \item usual suspects: union, intersection, subset \ldots
      \item >, >=, <, <=, in, \ldots
    \end{itemize}
    \begin{lstlisting}

In []: f10 = set([1,2,3,5,8])

In []: p10 = set([2,3,5,7])

In []: f10 | p10
Out[]: set([1, 2, 3, 5, 7, 8])
\end{lstlisting}
\end{frame}

\begin{frame}[fragile]
  \frametitle{Set \ldots}
    \begin{lstlisting}
In []: f10 & p10
Out[]: set([2, 3, 5])

In []: f10 - p10
Out[]: set([1, 8])

In []: p10 - f10, f10 ^ p10
Out[]: (set([7]), set([1, 7, 8]))

In []: set([2,3]) < p10
Out[]: True
\end{lstlisting}
\end{frame}

\begin{frame}[fragile]
  \frametitle{Set \ldots}
    \begin{lstlisting}
In []: set([2,3]) <= p10
Out[]: True

In []: 2 in p10
Out[]: True

In []: 4 in p10
Out[]: False

In []: len(f10)
Out[]: 5
\end{lstlisting}
\end{frame}

\begin{frame}
  \frametitle{Problem set 2.2: Problem 2.2.1}
Given a dictionary of the names of students and their marks, identify how many duplicate marks are there? and what are these?
\end{frame}

\begin{frame}
  \frametitle{Problem 2.2.2}
Given a list of words, find all the anagrams in the list.

\end{frame}

\section{Functions}
\begin{frame}[fragile]
  \frametitle{Functions}
  \begin{itemize}
    \item \kwrd{def} - keyword to define a function
    \item Arguments are local to a function
    \item Functions can return multiple values
  \end{itemize}
\end{frame}

\begin{frame}[fragile]
  \frametitle{Functions: example}
  \begin{lstlisting}
def signum( r ):
    """returns 0 if r is zero
    -1 if r is negative
    +1 if r is positive"""
    if r < 0:
        return -1
    elif r > 0:
        return 1
    else:
        return 0
  \end{lstlisting}
  \emphbar{Note docstrings}
\end{frame}

\begin{frame}[fragile]
  \frametitle {What does this function do?}
  \begin{lstlisting}
def what( n ):
    if n < 0: n = -n
    while n > 0:
        if n % 2 == 1:
            return False
        n /= 10
    return True
  \end{lstlisting}
\end{frame} 

\begin{frame}[fragile]
  {What does this function do?}
\begin{lstlisting}
def what( n ):
    i = 1
    while i * i < n:
        i += 1
    return i * i == n, i
  \end{lstlisting}
\end{frame}

\begin{frame}[fragile]
  \frametitle {What does this function do?}
  \begin{lstlisting}
def what( x, n ):
    if n < 0: 
       n = -n
       x = 1.0 / x

    z = 1.0
    while n > 0:
        if n % 2 == 1:
            z *= x
        x *= x
        n /= 2
    return z
  \end{lstlisting}
\end{frame} 

\begin{frame}
  \frametitle{What did we learn?}
  \begin{itemize}
    \item Loops: \kwrd{while}, \kwrd{for}
    \item Advanced Data structures:
    \begin{itemize}
      \item Lists
      \item Tuples
      \item Dictionaries
      \item Sets
    \end{itemize}
    \item Functions
    \item Docstrings
  \end{itemize}
\end{frame}

\end{document}

%% Questions for Quiz %%
%% ------------------ %%

\begin{frame}
    \frametitle{\incqno}
  How do you find the presence of an element \emph{x} in the list \emph{a}?
\end{frame}


\begin{frame}[fragile]
    \frametitle{\incqno}
  \begin{lstlisting}
In []: set([1, 2, 8, 2, 13, 8, 9])
  \end{lstlisting}
  What is the output?
\end{frame}

\begin{frame}[fragile]
    \frametitle{\incqno}
  \begin{lstlisting}
In []: a = {'a': 1, 'b': 2} 
In []: a['a'] = 10
In []: print a
  \end{lstlisting}
  What is the output?
\end{frame}

\begin{frame}[fragile]
    \frametitle{\incqno}
  \begin{lstlisting}
In []: for i in range(3, 10, 2):
  ...:     print i
  \end{lstlisting}
  What is the output?
\end{frame}

\begin{frame}[fragile]
    \frametitle{\incqno}
  \begin{lstlisting}
In []: a = [1, 2, 3] 
In []: a.extend([5, 6])
  \end{lstlisting}
  What is the value of a?
\end{frame}

\begin{frame}[fragile]
    \frametitle{\incqno}
  \begin{lstlisting}
In []: a = (1, 2, 3)
In []: a[1] = 10
  \end{lstlisting}
  What is the result?
\end{frame}

\begin{frame}[fragile]
    \frametitle{\incqno}
  \begin{lstlisting}
def func(x, y=10):
    print x+1, y+10

func(1)

  \end{lstlisting}
  What is the output?
\end{frame}

