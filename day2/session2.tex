%%%%%%%%%%%%%%%%%%%%%%%%%%%%%%%%%%%%%%%%%%%%%%%%%%%%%%%%%%%%%%%%%%%%%%%%%%%%%%%%
%Tutorial slides on Python.
%
% Author: Prabhu Ramachandran <prabhu at aero.iitb.ac.in>
% Copyright (c) 2005-2009, Prabhu Ramachandran
%%%%%%%%%%%%%%%%%%%%%%%%%%%%%%%%%%%%%%%%%%%%%%%%%%%%%%%%%%%%%%%%%%%%%%%%%%%%%%%%

\documentclass[14pt,compress]{beamer}
%\documentclass[draft]{beamer}
%\documentclass[compress,handout]{beamer}
%\usepackage{pgfpages} 
%\pgfpagesuselayout{2 on 1}[a4paper,border shrink=5mm]

% Modified from: generic-ornate-15min-45min.de.tex
\mode<presentation>
{
  \usetheme{Warsaw}
  \useoutertheme{split}
  \setbeamercovered{transparent}
}

\usepackage[english]{babel}
\usepackage[latin1]{inputenc}
%\usepackage{times}
\usepackage[T1]{fontenc}

% Taken from Fernando's slides.
\usepackage{ae,aecompl}
\usepackage{mathpazo,courier,euler}
\usepackage[scaled=.95]{helvet}

\definecolor{darkgreen}{rgb}{0,0.5,0}

\usepackage{listings}
\lstset{language=Python,
    basicstyle=\ttfamily\bfseries,
    commentstyle=\color{red}\itshape,
  stringstyle=\color{darkgreen},
  showstringspaces=false,
  keywordstyle=\color{blue}\bfseries}

%%%%%%%%%%%%%%%%%%%%%%%%%%%%%%%%%%%%%%%%%%%%%%%%%%%%%%%%%%%%%%%%%%%%%%
% Macros
\setbeamercolor{emphbar}{bg=blue!20, fg=black}
\newcommand{\emphbar}[1]
{\begin{beamercolorbox}[rounded=true]{emphbar} 
      {#1}
 \end{beamercolorbox}
}
\newcounter{time}
\setcounter{time}{0}
\newcommand{\inctime}[1]{\addtocounter{time}{#1}{\tiny \thetime\ m}}

\newcommand{\typ}[1]{\texttt{#1}}

\newcommand{\kwrd}[1]{ \texttt{\textbf{\color{blue}{#1}}}  }

%%% This is from Fernando's setup.
% \usepackage{color}
% \definecolor{orange}{cmyk}{0,0.4,0.8,0.2}
% % Use and configure listings package for nicely formatted code
% \usepackage{listings}
% \lstset{
%    language=Python,
%    basicstyle=\small\ttfamily,
%    commentstyle=\ttfamily\color{blue},
%    stringstyle=\ttfamily\color{orange},
%    showstringspaces=false,
%    breaklines=true,
%    postbreak = \space\dots
% }


%%%%%%%%%%%%%%%%%%%%%%%%%%%%%%%%%%%%%%%%%%%%%%%%%%%%%%%%%%%%%%%%%%%%%%
% Title page
\title[Basic Python]{Python:\\A formal approach}

\author[FOSSEE Team] {The FOSSEE Group}

\institute[IIT Bombay] {Department of Aerospace Engineering\\IIT Bombay}
\date[] {1, November 2009\\Day 2, Session 2}
%%%%%%%%%%%%%%%%%%%%%%%%%%%%%%%%%%%%%%%%%%%%%%%%%%%%%%%%%%%%%%%%%%%%%%

%\pgfdeclareimage[height=0.75cm]{iitmlogo}{iitmlogo}
%\logo{\pgfuseimage{iitmlogo}}


%% Delete this, if you do not want the table of contents to pop up at
%% the beginning of each subsection:
\AtBeginSubsection[]
{
  \begin{frame}<beamer>
    \frametitle{Outline}
    \tableofcontents[currentsection,currentsubsection]
  \end{frame}
}

\AtBeginSection[]
{
  \begin{frame}<beamer>
    \frametitle{Outline}
    \tableofcontents[currentsection,currentsubsection]
  \end{frame}
}

% If you wish to uncover everything in a step-wise fashion, uncomment
% the following command: 
%\beamerdefaultoverlayspecification{<+->}

%\includeonlyframes{current,current1,current2,current3,current4,current5,current6}

%%%%%%%%%%%%%%%%%%%%%%%%%%%%%%%%%%%%%%%%%%%%%%%%%%%%%%%%%%%%%%%%%%%%%%
% DOCUMENT STARTS
\begin{document}

\begin{frame}
  \titlepage
\end{frame}

\begin{frame}
  \frametitle{Outline}
  \tableofcontents
  % You might wish to add the option [pausesections]
\end{frame}

\section{Data structures}
\subsection{Lists}
\begin{frame}[fragile]
  \frametitle{Lists}
\begin{block}{We already know that}
  \begin{lstlisting}
num = [1, 2, 3, 4, 5, 6, 7, 8]
  \end{lstlisting}
\centerline{is a list}
\end{block}
\end{frame}

\begin{frame}[fragile]
  \frametitle{Lists: methods}
  \begin{lstlisting}
In []: num.reverse()
In []: num
Out[]: [8, 7, 6, 5, 4, 3, 2, 1]

In []: num.extend([0, -1, -2])
In []: num
Out[]: [8, 7, 6, 5, 4, 3, 2, 1, 0, -1]

In []: num.remove(0)
In []: num
Out[]: [8, 7, 6, 5, 4, 3, 2, 1, -1]
  \end{lstlisting}
\end{frame}

\begin{frame}[fragile]
\frametitle{List containership}
\begin{lstlisting}
In []: a = 8

In []: a in num
Out[]: True

In []: b = 10
In []: b in num
Out[]: False

In []: b not in num
Out[]: True
\end{lstlisting}
\end{frame}

\subsection{Tuples}
\begin{frame}[fragile]
\frametitle{Tuples: Immutable lists}
\begin{lstlisting}
In []: t = (1, 2, 3, 4, 5, 6, 7, 8)
In []: t[0] + t[3] + t[-1]
Out[]: 13
\end{lstlisting}
\begin{block}{Note:}
\begin{itemize}
  \item Tuples are immutable - cannot be changed
\end{itemize}
\end{block}
  \inctime{10}
\end{frame}

\begin{frame}
  {A classic problem}
  \begin{block}
    {Interchange values}
    How to interchange values of two variables? 
  \end{block}
  \pause
  \begin{block}{Note:}
    This Python idiom works for all types of variables.\\
They need not be of the same type!
  \end{block}
\end{frame}

\subsection{Dictionaries}
\begin{frame}[fragile]
  \frametitle{Dictionaries: Recall}
  \begin{lstlisting}
In []: player = {'Mat': 134,'Inn': 233,
          'Runs': 10823, 'Avg': 52.53}

In []: player['Avg']
Out[]: 52.530000000000001
  \end{lstlisting}
  \begin{block}{Note!}
    Duplicate keys are not allowed!\\
    Dictionaries are iterable through keys.
  \end{block}
\end{frame}

\begin{frame} {Problem Set 2.1: Problem 2.1.1}
You are given date strings of the form ``29, Jul 2009'', or ``4 January 2008''. In other words a number a string and another number, with a comma sometimes separating the items.Write a function that takes such a string and returns a tuple (yyyy, mm, dd) where all three elements are ints.
\end{frame}

\subsection{Set}
\begin{frame}[fragile]
  \frametitle{Set}
    \begin{itemize}
      \item Simplest container, mutable
      \item No ordering, no duplicates
      \item usual suspects: union, intersection, subset \ldots
      \item >, >=, <, <=, in, \ldots
    \end{itemize}
    \begin{lstlisting}
>>> f10 = set([1,2,3,5,8])
>>> p10 = set([2,3,5,7])
>>> f10|p10
set([1, 2, 3, 5, 7, 8])
>>> f10&p10
set([2, 3, 5])
>>> f10-p10
set([8, 1])
\end{lstlisting}
\end{frame}

\begin{frame}[fragile]
  \frametitle{Set}
    \begin{lstlisting}
>>> p10-f10, f10^p10
set([7]), set([1, 7, 8])
>>> set([2,3]) < p10
True
>>> set([2,3]) <= p10
True
>>> 2 in p10
True
>>> 4 in p10
False
>>> len(f10)
5
\end{lstlisting}
\end{frame}

\begin{frame}
  \frametitle{Problem set 2.2}
  \begin{description}
    \item[2.2.1] Given a dictionary of the names of students and their marks, identify how many duplicate marks are there? and what are these?
    \item[2.2.2] Given a string of the form ``4-7, 9, 12, 15'' find the numbers missing in this list for a given range.
\end{description}
\inctime{15}
\end{frame}

\section{Functions}
\begin{frame}[fragile]
  \frametitle{Functions}
  \begin{itemize}
    \item \kwrd{def} - keyword to define a function
    \item Arguments are local to a function
    \item Docstrings are important!
    \item Functions can return multiple values
  \end{itemize}
\end{frame}

\begin{frame}[fragile]
  \frametitle{Functions: example}
  \begin{lstlisting}
def signum( r ):
    """returns 0 if r is zero
    -1 if r is negative
    +1 if r is positive"""
    if r < 0:
        return -1
    elif r > 0:
        return 1
    else:
        return 0
  \end{lstlisting}
\end{frame}

\begin{frame}[fragile]
  {What does this function do?}
\begin{lstlisting}
def what( n ):
    i = 1
    while i * i < n:
        i += 1
    return i * i == n, i
  \end{lstlisting}
\end{frame}

\subsection{Default arguments}
\begin{frame}[fragile]
  \frametitle{Functions: default arguments}
  \small
  \begin{lstlisting}
def ask_ok(prompt, complaint='Yes or no!'):
    while True:
        ok = raw_input(prompt)
        if ok in ('y', 'ye', 'yes'): 
            return True
        if ok in ('n', 'no', 'nop',
                  'nope'): 
            return False
        print complaint

ask_ok('?')
ask_ok('?', '[Y/N]')
  \end{lstlisting}
\end{frame}

\subsection{Built-in functions}
\begin{frame}
  {Before writing a function}
  \begin{itemize}
      \item Variety of builtin functions are available
      \item \typ{abs, any, all, len, max, min}
      \item \typ{pow, range, sum, type}
      \item Refer here:
          \url{http://docs.python.org/library/functions.html}
  \end{itemize}
  \inctime{10} 
\end{frame}

\subsection{Exercises}
\begin{frame}{Problem set 3: Problem 3.1}
  Write a function to return the gcd of two numbers.
\end{frame}

\begin{frame}{Problem 3.2}
Write a program to print all primitive pythagorean triads (a, b, c) where a, b are in the range 1---100 \\
A pythagorean triad $(a,b,c)$ has the property $a^2 + b^2 = c^2$.\\By primitive we mean triads that do not `depend' on others. For example, (4,3,5) is a variant of (3,4,5) and hence is not primitive. And (10,24,26) is easily derived from (5,12,13) and is also not primitive.
\end{frame}

\begin{frame}{Problem 3.3}
  Write a program that generates a list of all four digit numbers that have all their digits even and are perfect squares.\newline\\\emph{For example, the output should include 6400 but not 8100 (one digit is odd) or 4248 (not a perfect square).}
\inctime{15}
\end{frame}

\section{Modules}
\begin{frame}[fragile]
    {Modules}
\begin{lstlisting}
>>> sqrt(2)
Traceback (most recent call last):
  File "<stdin>", line 1, in <module>
NameError: name 'sqrt' is not defined
>>> import math        
>>> math.sqrt(2)
1.4142135623730951
\end{lstlisting}
\end{frame}

\begin{frame}[fragile]
    {Modules}
  \begin{itemize}
    \item The \kwrd{import} keyword ``loads'' a module
    \item One can also use:
      \begin{lstlisting}
>>> from math import sqrt
>>> from math import *
      \end{lstlisting}    
    \item What is the difference?
    \item \alert{Use the latter only in interactive mode}
    \end{itemize}
  \emphbar{Package hierarchies}
      \begin{lstlisting}
>>> from os.path import exists
      \end{lstlisting}
\end{frame}

\begin{frame}
  \frametitle{Modules: Standard library}
  \begin{itemize}
  \item Very powerful, ``Batteries included''
  \item Some standard modules:
    \begin{itemize}
    \item Math: \typ{math}, \typ{random}
    \item Internet access: \typ{urllib2}, \typ{smtplib}
    \item System, Command line arguments: \typ{sys}
    \item Operating system interface: \typ{os}
    \item Regular expressions: \typ{re}
    \item Compression: \typ{gzip}, \typ{zipfile}, and \typ{tarfile}
    \item And a whole lot more!
    \end{itemize}
  \item Check out the Python Library reference:
    \url{http://docs.python.org/library/}
  \end{itemize}
\inctime{5}
\end{frame}

\section{Coding Style}
\begin{frame}{Readability and Consistency}
    \begin{itemize}
        \item Readability Counts!\\Code is read more often than its written.
        \item Consistency!
        \item Know when to be inconsistent.
      \end{itemize}
\end{frame}

\begin{frame}[fragile] \frametitle{A question of good style}
  \begin{lstlisting}
    amount = 12.68
    denom = 0.05
    nCoins = round(amount/denom)
    rAmount = nCoins * denom
  \end{lstlisting}
  \pause
  \begin{block}{Style Rule \#1}
    Naming is 80\% of programming
  \end{block}
\end{frame}

\begin{frame}[fragile]
  \frametitle{Code Layout}
  \begin{itemize}
        \item Indentation
        \item Tabs or Spaces??
        \item Maximum Line Length
        \item Blank Lines
        \item Encodings
   \end{itemize}
\end{frame}

\begin{frame}{Whitespaces in Expressions}
  \begin{itemize}
        \item When to use extraneous whitespaces??
        \item When to avoid extra whitespaces??
        \item Use one statement per line
   \end{itemize}
\end{frame}

\begin{frame}{Comments}
  \begin{itemize}
        \item No comments better than contradicting comments
        \item Block comments
        \item Inline comments
   \end{itemize}
\end{frame}

\begin{frame}{Docstrings}
  \begin{itemize}
        \item When to write docstrings?
        \item Ending the docstrings
        \item One liner docstrings
   \end{itemize}
More information at PEP8: http://www.python.org/dev/peps/pep-0008/
\inctime{5}
\end{frame}

\section{Objects}
\begin{frame}{Objects in general}
    \begin{itemize}
        \item What is an Object? (Types and classes)
        \item identity
        \item type
        \item method
      \end{itemize}
\end{frame}

\begin{frame}{Almost everything is an Object!}
  \begin{itemize}
    \item \typ{list}
    \item \typ{tuple}
    \item \typ{string}
    \item \typ{dictionary}
    \item \typ{function}
    \item Of course, user defined class objects!
  \end{itemize}
\end {frame}

\begin{frame}{Using Objects}
  \begin{itemize}
    \item Creating Objects: Initialization
    \item Object Manipulation: Object methods and ``.'' operator
  \end{itemize}
\end{frame}

\begin{frame}[fragile]
  \frametitle{Objects provide consistency}
  \small
  \begin{lstlisting}
for element in (1, 2, 3):
    print element
for key in {'one':1, 'two':2}:
    print key
for char in "123":
    print char
for line in open("myfile.txt"):
    print line
for line in urllib2.urlopen('http://site.com'):
    print line
  \end{lstlisting}
\inctime{10}
\end{frame}

\begin{frame}
  \frametitle{What did we learn?}
  \begin{itemize}
    \item Lists, Tuples, Dictionaries, Sets: creation and manipulation
    \item More about functions
    \item Coding style
    \item Objects: creation and manipulation
  \end{itemize}
\end{frame}

\end{document}
