%%%%%%%%%%%%%%%%%%%%%%%%%%%%%%%%%%%%%%%%%%%%%%%%%%%%%%%%%%%%%%%%%%%%%%%%%%%%%%%%
% Tutorial slides on Python.
%
% Author: Prabhu Ramachandran <prabhu at aero.iitb.ac.in>
% Copyright (c) 2005-2008, Prabhu Ramachandran
%%%%%%%%%%%%%%%%%%%%%%%%%%%%%%%%%%%%%%%%%%%%%%%%%%%%%%%%%%%%%%%%%%%%%%%%%%%%%%%%

\documentclass[14pt,compress]{beamer}
%\documentclass[draft]{beamer}
%\documentclass[compress,handout]{beamer}
%\usepackage{pgfpages} 
%\pgfpagesuselayout{2 on 1}[a4paper,border shrink=5mm]

% Modified from: generic-ornate-15min-45min.de.tex
\mode<presentation>
{
  \usetheme{Warsaw}
  \useoutertheme{split}
  \setbeamercovered{transparent}
}

\usepackage[english]{babel}
\usepackage[latin1]{inputenc}
%\usepackage{times}
\usepackage[T1]{fontenc}

% Taken from Fernando's slides.
\usepackage{ae,aecompl}
\usepackage{mathpazo,courier,euler}
\usepackage[scaled=.95]{helvet}

\definecolor{darkgreen}{rgb}{0,0.5,0}

\usepackage{listings}
\lstset{language=Python,
    basicstyle=\ttfamily\bfseries,
    commentstyle=\color{red}\itshape,
  stringstyle=\color{darkgreen},
  showstringspaces=false,
  keywordstyle=\color{blue}\bfseries}

%%%%%%%%%%%%%%%%%%%%%%%%%%%%%%%%%%%%%%%%%%%%%%%%%%%%%%%%%%%%%%%%%%%%%%
% Macros
\setbeamercolor{emphbar}{bg=blue!20, fg=black}
\newcommand{\emphbar}[1]
{\begin{beamercolorbox}[rounded=true]{emphbar} 
      {#1}
 \end{beamercolorbox}
}
\newcounter{time}
\setcounter{time}{0}
\newcommand{\inctime}[1]{\addtocounter{time}{#1}{\tiny \thetime\ m}}

\newcommand{\typ}[1]{\lstinline{#1}}

\newcommand{\kwrd}[1]{ \texttt{\textbf{\color{blue}{#1}}}  }

%%% This is from Fernando's setup.
% \usepackage{color}
% \definecolor{orange}{cmyk}{0,0.4,0.8,0.2}
% % Use and configure listings package for nicely formatted code
% \usepackage{listings}
% \lstset{
%    language=Python,
%    basicstyle=\small\ttfamily,
%    commentstyle=\ttfamily\color{blue},
%    stringstyle=\ttfamily\color{orange},
%    showstringspaces=false,
%    breaklines=true,
%    postbreak = \space\dots
% }


%%%%%%%%%%%%%%%%%%%%%%%%%%%%%%%%%%%%%%%%%%%%%%%%%%%%%%%%%%%%%%%%%%%%%%
% Title page
\title[]{Numerical Computing with Numpy \& Scipy}

\author[FOSSEE Team] {Asokan Pichai\\Prabhu Ramachandran}

\institute[FOSSEE] {FOSSEE Team}
\date[] {11, October 2009}
%%%%%%%%%%%%%%%%%%%%%%%%%%%%%%%%%%%%%%%%%%%%%%%%%%%%%%%%%%%%%%%%%%%%%%

%\pgfdeclareimage[height=0.75cm]{iitmlogo}{iitmlogo}
%\logo{\pgfuseimage{iitmlogo}}


%% Delete this, if you do not want the table of contents to pop up at
%% the beginning of each subsection:
\AtBeginSubsection[]
{
  \begin{frame}<beamer>
    \frametitle{Outline}
    \tableofcontents[currentsection,currentsubsection]
  \end{frame}
}

\AtBeginSection[]
{
  \begin{frame}<beamer>
    \frametitle{Outline}
    \tableofcontents[currentsection,currentsubsection]
  \end{frame}
}

% If you wish to uncover everything in a step-wise fashion, uncomment
% the following command: 
%\beamerdefaultoverlayspecification{<+->}

%\includeonlyframes{current,current1,current2,current3,current4,current5,current6}

%%%%%%%%%%%%%%%%%%%%%%%%%%%%%%%%%%%%%%%%%%%%%%%%%%%%%%%%%%%%%%%%%%%%%%
% DOCUMENT STARTS
\begin{document}

\begin{frame}[fragile]
  \frametitle{Broadcasting}

\end{frame}

\begin{frame}[fragile]
  \frametitle{Copies \& Views}

\end{frame}

\begin{frame}
  \frametitle{More Numpy Functions \& Methods}
  More functions
  \begin{itemize}
    \item \typ{take}
    \item \typ{choose}
    \item \typ{where}
    \item \typ{compress}
    \item \typ{concatenate}
  \end{itemize}
  Ufunc methods 
  \begin{itemize}
    \item \typ{reduce}
    \item \typ{accumulate}
    \item \typ{outer}
    \item \typ{reduceat}
  \end{itemize}
\end{frame}

\begin{frame}
    {Intro to SciPy}
  \begin{itemize}
  \item \url{http://www.scipy.org}
  \item Open source scientific libraries for Python
  \item Based on NumPy
    \end{itemize}

    \inctime{25}
\end{frame}

\begin{frame}
  \frametitle{SciPy}
  \begin{itemize}
  \item Provides:
    \begin{itemize}
    \item Linear algebra
    \item Numerical integration
    \item Fourier transforms
    \item Signal processing
    \item Special functions
    \item Statistics
    \item Optimization
    \item Image processing
    \item ODE solvers
    \end{itemize}
  \item Uses LAPACK, QUADPACK, ODEPACK, FFTPACK etc. from netlib
  \end{itemize}
\end{frame}

\begin{frame}[fragile]
  \frametitle{Linear Algebra}

\end{frame}
\begin{frame}[fragile]
  \frametitle{ODEs}

\end{frame}
\begin{frame}[fragile]
  \frametitle{Interpolation}

\end{frame}

\begin{frame}[fragile]
  \frametitle{Signal \& Image Processing}

\end{frame}

\begin{frame}[fragile]
  \frametitle{Problems}

\end{frame}


\end{document}

- Numpy arrays (30 mins)
    - Matrices
    - random number generation.
    - Image manipulation: jigsaw puzzle.
    - Monte-carlo integration.


\subsection{SciPy}

\begin{frame}
  \frametitle{Using \texttt{SciPy}}
  \begin{itemize}
  \item SciPy is Open Source software for mathematics, science, and
    engineering
  \item \typ{import scipy}
  \item Built on NumPy
  \item Provides modules for statistics, optimization, integration,
    linear algebra, Fourier transforms, signal and image processing,
    genetic algorithms, ODE solvers, special functions, and more
  \item Used widely by scientists world over
  \item Details are beyond the scope of this tutorial
  \end{itemize}
\end{frame}

