%%%%%%%%%%%%%%%%%%%%%%%%%%%%%%%%%%%%%%%%%%%%%%%%%%%%%%%%%%%%%%%%%%%%%%%%%%%%%%%%
% Tutorial slides on Python.
%
% Author: Prabhu Ramachandran <prabhu at aero.iitb.ac.in>
% Copyright (c) 2005-2009, Prabhu Ramachandran
%%%%%%%%%%%%%%%%%%%%%%%%%%%%%%%%%%%%%%%%%%%%%%%%%%%%%%%%%%%%%%%%%%%%%%%%%%%%%%%%

\documentclass[compress,14pt]{beamer}
% \documentclass[handout]{beamer}
% \usepackage{pgfpages}
% \pgfpagesuselayout{4 on 1}[a4paper,border, shrink=5mm,landscape]
\usepackage{tikz}
\newcommand{\hyperlinkmovie}{}
%\usepackage{movie15}

%%%%%%%%%%%%%%%%%%%%%%%%%%%%%%%%%%%%%%%%%%%%%%%%%%%%%%%%%%%%%%%%%%%%%%%%%%%%%%
% Note that in presentation mode 
% \paperwidth  364.19536pt
% \paperheight 273.14662pt
% h/w = 0.888


\mode<presentation>
{
  \usetheme{Warsaw}
  %\usetheme{Boadilla}
  %\usetheme{default}
  \useoutertheme{infolines}
  \setbeamercovered{transparent}
}

% To remove navigation symbols
\setbeamertemplate{navigation symbols}{}

\usepackage{amsmath}
\usepackage[english]{babel}
\usepackage[latin1]{inputenc}
\usepackage{times}
\usepackage[T1]{fontenc}

% Taken from Fernando's slides.
\usepackage{ae,aecompl}
\usepackage{mathpazo,courier,euler}
\usepackage[scaled=.95]{helvet}
\usepackage{pgf}

\definecolor{darkgreen}{rgb}{0,0.5,0}

\usepackage{listings}
\lstset{language=Python,
    basicstyle=\ttfamily\bfseries,
    commentstyle=\color{red}\itshape,
  stringstyle=\color{darkgreen},
  showstringspaces=false,
  keywordstyle=\color{blue}\bfseries}

%%%%%%%%%%%%%%%%%%%%%%%%%%%%%%%%%%%%%%%%%%%%%%%%%%%%%%%%%%%%%%%%%%%%%%
% My Macros
\setbeamercolor{postit}{bg=yellow,fg=black}
\setbeamercolor{emphbar}{bg=blue!20, fg=black}
\newcommand{\emphbar}[1]
{\begin{beamercolorbox}[rounded=true]{emphbar} 
      {#1}
 \end{beamercolorbox}
}
%{\centerline{\fcolorbox{gray!50} {blue!10}{
%\begin{minipage}{0.9\linewidth}
%    {#1} 
%\end{minipage}
%    }}}

\newcommand{\myemph}[1]{\structure{\emph{#1}}}
\newcommand{\PythonCode}[1]{\lstinline{#1}}

\newcommand{\tvtk}{\texttt{tvtk}}
\newcommand{\mlab}{\texttt{mlab}}

\newcounter{time}
\setcounter{time}{0}
\newcommand{\inctime}[1]{\addtocounter{time}{#1}{\vspace*{0.1in}\tiny \thetime\ m}}

\newcommand\BackgroundPicture[1]{%
  \setbeamertemplate{background}{%
      \parbox[c][\paperheight]{\paperwidth}{%
      \vfill \hfill
 \hfill \vfill
}}}

%%%%%%%%%%%%%%%%%%%%%%%%%%%%%%%%%%%%%%%%%%%%%%%%%%%%%%%%%%%%%%%%%%%%%%
% Configuring the theme
%\setbeamercolor{normal text}{fg=white}
%\setbeamercolor{background canvas}{bg=black}



%%%%%%%%%%%%%%%%%%%%%%%%%%%%%%%%%%%%%%%%%%%%%%%%%%%%%%%%%%%%%%%%%%%%%%
% Title page
\title[]{Debugging and \\Test Driven Approach}

\author[FOSSEE] {FOSSEE}

\institute[IIT Bombay] {Department of Aerospace Engineering\\IIT Bombay}
\date[] {11, October 2009}
\date[] % (optional)

%%%%%%%%%%%%%%%%%%%%%%%%%%%%%%%%%%%%%%%%%%%%%%%%%%%%%%%%%%%%%%%%%%%%%%

%\pgfdeclareimage[height=0.75cm]{iitblogo}{iitblogo}
%\logo{\pgfuseimage{iitblogo}}

\AtBeginSection[]
{
  \begin{frame}<beamer>
    \frametitle{Outline}
      \Large
    \tableofcontents[currentsection,currentsubsection]
  \end{frame}
}

%% Delete this, if you do not want the table of contents to pop up at
%% the beginning of each subsection:
\AtBeginSubsection[]
{
  \begin{frame}<beamer>
    \frametitle{Outline}
    \tableofcontents[currentsection,currentsubsection]
  \end{frame}
}

\AtBeginSection[]
{
  \begin{frame}<beamer>
    \frametitle{Outline}
    \tableofcontents[currentsection,currentsubsection]
  \end{frame}
}
%%%%%%%%%%%%%%%%%%%%%%%%%%%%%%%%%%%%%%%%%%%%%%%%%%%%%%%%%%%%%%%%%%%%%%
% DOCUMENT STARTS
\begin{document}

\begin{frame}
  \maketitle
\end{frame}


\section{Debugging}
\subsection{Errors and Exceptions}
\begin{frame}[fragile]
 \frametitle{Errors}
 \begin{lstlisting}
>>> while True print 'Hello world'
 \end{lstlisting}
\pause
  \begin{lstlisting}
  File "<stdin>", line 1, in ?
    while True print 'Hello world'
                   ^
SyntaxError: invalid syntax
\end{lstlisting}
\end{frame}

\begin{frame}[fragile]
 \frametitle{Exceptions}
 \begin{lstlisting}
>>> print spam
\end{lstlisting}
\pause
\begin{lstlisting}
Traceback (most recent call last):
  File "<stdin>", line 1, in <module>
NameError: name 'spam' is not defined
\end{lstlisting}
\end{frame}

\begin{frame}[fragile]
 \frametitle{Exceptions}
 \begin{lstlisting}
>>> 1 / 0
\end{lstlisting}
\pause
\begin{lstlisting}
Traceback (most recent call last):
  File "<stdin>", line 1, in <module>
ZeroDivisionError: integer division 
or modulo by zero
\end{lstlisting}
\end{frame}

\subsection{Strategy}
\begin{frame}[fragile]
    \frametitle{Debugging effectively}
    \begin{itemize}
        \item \kwrd{print} based strategy
        \item Process:
    \end{itemize}
\begin{center}
\pgfimage[interpolate=true,width=5cm,height=5cm]{DebugginDiagram.png}
\end{center}
\end{frame}

\begin{frame}[fragile]
    \frametitle{Debugging effectively}
    \begin{itemize}
      \item Using \typ{\%debug} in IPython
    \end{itemize}
\end{frame}

\begin{frame}[fragile]
\frametitle{Debugging in IPython}
\small
\begin{lstlisting}
In [1]: import mymodule
In [2]: mymodule.test()
---------------------------------------------
NameError   Traceback (most recent call last)
<ipython console> in <module>()
mymodule.py in test()
      1 def test():
----> 2     print spam
NameError: global name 'spam' is not defined

In [3]: %debug
> mymodule.py(2)test()
      0     print spam
ipdb> 
\end{lstlisting}
\inctime{15} 
\end{frame}

\subsection{Exercise}
\begin{frame}[fragile]
\frametitle{Debugging: Exercise}
\small
\begin{lstlisting}
import keyword
f = open('/path/to/file')

freq = {}
for line in f:
    words = line.split()
    for word in words:
        key = word.strip(',.!;?()[]: ')
        if keyword.iskeyword(key):
            value = freq[key]
            freq[key] = value + 1

print freq
\end{lstlisting}
\inctime{10}
\end{frame}

%% \begin{frame}
%%     \frametitle{Testing}
   
%%     \begin{itemize}
%%         \item Writing tests is really simple!

%%         \item Using nose.

%%         \item Example!
%%     \end{itemize}
%% \end{frame}

\section{Test Driven Approach}
\begin{frame}
    \frametitle{Need of Testing!}
   
    \begin{itemize}
        \item Quality
        \item Regression
        \item Documentation
    \end{itemize}
    %% \vspace*{0.25in}
    %% \emphbar{It is to assure that section of code is working as it is supposed to work}
\end{frame}

\begin{frame}[fragile]
    \frametitle{Example}
    \begin{block}{Problem Statement}
      Write a function to check whether a given input
      string is a palindrome.
    \end{block}
\end{frame}

\begin{frame}[fragile]
    \frametitle{Function: palindrome.py}
\begin{lstlisting}    
def is_palindrome(input_str):
  return input_str == input_str[::-1]
\end{lstlisting}    
\end{frame}

\begin{frame}[fragile]
    \frametitle{Test for the palindrome: palindrome.py}
\begin{lstlisting}    
from plaindrome import is_palindrome
def test_function_normal_words():
  input = "noon"
  assert is_palindrome(input) == True
\end{lstlisting}    
\end{frame}

\begin{frame}[fragile]
    \frametitle{Running the tests.}
\begin{lstlisting}    
$ nosetests test.py 
.
----------------------------------------------
Ran 1 test in 0.001s

OK
\end{lstlisting}    
\end{frame}

\begin{frame}[fragile]
    \frametitle{Exercise: Including new tests.}
\begin{lstlisting}    
def test_function_ignore_cases_words():
  input = "Noon"
  assert is_palindrome(input) == True
\end{lstlisting}
     \vspace*{0.25in}
     Check\\
     \PythonCode{$ nosetests test.py} \\
     \begin{block}{Task}
     Tweak the code to pass this test.
     \end{block}
\end{frame}

%\begin{frame}[fragile]
%    \frametitle{Lets write some test!}
%\begin{lstlisting}    
%#for form of equation y=mx+c
%#given m and c for two equation,
%#finding the intersection point.
%def intersect(m1,c1,m2,c2):
%    x = (c2-c1)/(m1-m2)
%    y = m1*x+c1
%    return (x,y)
%\end{lstlisting}
%
%Create a simple test for this
%
%function which will make it fail.
%
%\inctime{15} 
%\end{frame}
%

\begin{frame}[fragile]
    \frametitle{Exercise}
    Based on Euclid's algorithm:
    \begin{center}
    $gcd(a,b)=gcd(b,b\%a)$
    \end{center}
    gcd function can be written as:
    \begin{lstlisting}
    def gcd(a, b):
      if a%b == 0: return b
      return gcd(b, a%b)
    \end{lstlisting}
    \vspace*{-0.15in}
    \begin{block}{Task}
      \begin{itemize}
      \item Write at least 
        two tests for above mentioned function.
      \item Write a non recursive implementation
      of gcd(), and test it using already 
      written tests.
      \end{itemize}
    \end{block}
    
\inctime{15} 
\end{frame}

\begin{frame}
  \frametitle{We have learned}
  \begin{itemize}
  \item Following and Resolving Error Messages.
  \item Exceptions.
  \item Approach for Debugging.
  \item Writting and running tests.
  \end{itemize}
\end{frame}

\end{document}
