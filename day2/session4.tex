%%%%%%%%%%%%%%%%%%%%%%%%%%%%%%%%%%%%%%%%%%%%%%%%%%%%%%%%%%%%%%%%%%%%%%%%%%%%%%%%
% Tutorial slides on Python.
%
% Author: Prabhu Ramachandran <prabhu at aero.iitb.ac.in>
% Copyright (c) 2005-2009, Prabhu Ramachandran
%%%%%%%%%%%%%%%%%%%%%%%%%%%%%%%%%%%%%%%%%%%%%%%%%%%%%%%%%%%%%%%%%%%%%%%%%%%%%%%%

\documentclass[compress,14pt]{beamer}
% \documentclass[handout]{beamer}
% \usepackage{pgfpages}
% \pgfpagesuselayout{4 on 1}[a4paper,border, shrink=5mm,landscape]
\usepackage{tikz}
\newcommand{\hyperlinkmovie}{}
%\usepackage{movie15}

%%%%%%%%%%%%%%%%%%%%%%%%%%%%%%%%%%%%%%%%%%%%%%%%%%%%%%%%%%%%%%%%%%%%%%%%%%%%%%
% Note that in presentation mode 
% \paperwidth  364.19536pt
% \paperheight 273.14662pt
% h/w = 0.888


\mode<presentation>
{
  \usetheme{Warsaw}
  %\usetheme{Boadilla}
  %\usetheme{default}
  \useoutertheme{infolines}
  \setbeamercovered{transparent}
}

% To remove navigation symbols
\setbeamertemplate{navigation symbols}{}

\usepackage{amsmath}
\usepackage[english]{babel}
\usepackage[latin1]{inputenc}
\usepackage{times}
\usepackage[T1]{fontenc}

% Taken from Fernando's slides.
\usepackage{ae,aecompl}
\usepackage{mathpazo,courier,euler}
\usepackage[scaled=.95]{helvet}
\usepackage{pgf}

\definecolor{darkgreen}{rgb}{0,0.5,0}

\usepackage{listings}
\lstset{language=Python,
    basicstyle=\ttfamily\bfseries,
    commentstyle=\color{red}\itshape,
  stringstyle=\color{darkgreen},
  showstringspaces=false,
  keywordstyle=\color{blue}\bfseries}

%%%%%%%%%%%%%%%%%%%%%%%%%%%%%%%%%%%%%%%%%%%%%%%%%%%%%%%%%%%%%%%%%%%%%%
% My Macros
\setbeamercolor{postit}{bg=yellow,fg=black}
\setbeamercolor{emphbar}{bg=blue!20, fg=black}
\newcommand{\emphbar}[1]
{\begin{beamercolorbox}[rounded=true]{emphbar} 
      {#1}
 \end{beamercolorbox}
}
%{\centerline{\fcolorbox{gray!50} {blue!10}{
%\begin{minipage}{0.9\linewidth}
%    {#1} 
%\end{minipage}
%    }}}

\newcommand{\myemph}[1]{\structure{\emph{#1}}}
\newcommand{\PythonCode}[1]{\lstinline{#1}}

\newcommand{\tvtk}{\texttt{tvtk}}
\newcommand{\mlab}{\texttt{mlab}}
\newcommand{\typ}[1]{\lstinline{#1}}
\newcounter{time}
\setcounter{time}{0}
\newcommand{\inctime}[1]{\addtocounter{time}{#1}{\vspace*{0.1in}\tiny \thetime\ m}}

\newcommand\BackgroundPicture[1]{%
  \setbeamertemplate{background}{%
      \parbox[c][\paperheight]{\paperwidth}{%
      \vfill \hfill
 \hfill \vfill
}}}

%%%%%%%%%%%%%%%%%%%%%%%%%%%%%%%%%%%%%%%%%%%%%%%%%%%%%%%%%%%%%%%%%%%%%%
% Configuring the theme
%\setbeamercolor{normal text}{fg=white}
%\setbeamercolor{background canvas}{bg=black}



%%%%%%%%%%%%%%%%%%%%%%%%%%%%%%%%%%%%%%%%%%%%%%%%%%%%%%%%%%%%%%%%%%%%%%
% Title page
\title[Python Development]{Python Development}

\author[FOSSEE] {FOSSEE}

\institute[IIT Bombay] {Department of Aerospace Engineering\\IIT Bombay}
\date[] {12 January, 2010\\Day 2, Session 4}
%%%%%%%%%%%%%%%%%%%%%%%%%%%%%%%%%%%%%%%%%%%%%%%%%%%%%%%%%%%%%%%%%%%%%%

%\pgfdeclareimage[height=0.75cm]{iitblogo}{iitblogo}
%\logo{\pgfuseimage{iitblogo}}

\AtBeginSection[]
{
  \begin{frame}<beamer>
    \frametitle{Outline}
      \Large
    \tableofcontents[currentsection,currentsubsection]
  \end{frame}
}

%% Delete this, if you do not want the table of contents to pop up at
%% the beginning of each subsection:
\AtBeginSubsection[]
{
  \begin{frame}<beamer>
    \frametitle{Outline}
    \tableofcontents[currentsection,currentsubsection]
  \end{frame}
}

\AtBeginSection[]
{
  \begin{frame}<beamer>
    \frametitle{Outline}
    \tableofcontents[currentsection,currentsubsection]
  \end{frame}
}
%%%%%%%%%%%%%%%%%%%%%%%%%%%%%%%%%%%%%%%%%%%%%%%%%%%%%%%%%%%%%%%%%%%%%%
% DOCUMENT STARTS
\begin{document}

\begin{frame}
  \maketitle
\end{frame}

\section{Tests: Getting started}
\begin{frame}[fragile] 
  \frametitle{gcd revisited!}
  \begin{itemize}
  \item Open \texttt{gcd.py}
  \end{itemize}  
\begin{lstlisting}
    def gcd(a, b):
        if a % b == 0: 
            return b
        return gcd(b, a%b)

    print gcd(15, 65)
    print gcd(16, 76)
\end{lstlisting}
  \begin{itemize}
  \item \texttt{python gcd.py}
  \end{itemize}
\end{frame}

\begin{frame}[fragile] 
  \frametitle{Find lcm using our gcd module}
  \begin{itemize}
  \item Open \texttt{lcm.py}  
  \item $lcm = \frac{a * b}{gcd(a,b)}$
  \end{itemize}  
\begin{lstlisting}
    from gcd import gcd    
    def lcm(a, b):
        return (a * b) / gcd(a, b)
    
    print lcm(14, 56)
\end{lstlisting}
  \begin{itemize}
  \item \texttt{python lcm.py}
  \end{itemize}
  \begin{lstlisting}
5
4
56
  \end{lstlisting}    
\end{frame}

\begin{frame}[fragile] 
  \frametitle{Writing stand-alone module}  
Edit \texttt{gcd.py} file to:
\begin{lstlisting}
    def gcd(a, b):
        if a % b == 0: 
            return b
        return gcd(b, a%b)

    if __name__ == "__main__":        
        print gcd(15, 65)
        print gcd(16, 76)
\end{lstlisting}
  \begin{itemize}
  \item \texttt{python gcd.py}
  \item \texttt{python lcm.py}
  \end{itemize}
\end{frame}

\begin{frame}[fragile]
  \frametitle{Automating tests}
  \begin{lstlisting}
if __name__ == '__main__':
    for line in open('numbers.txt'):
        numbers = line.split()
        x = int(numbers[0])
        y = int(numbers[1])
        result = int(numbers[2])
        if gcd(x, y) != result:
            print "Failed gcd test
                          for", x, y
  \end{lstlisting}  
\end{frame}

\section{Coding Style}
\begin{frame}{Readability and Consistency}
    \begin{itemize}
        \item Readability Counts!\\Code is read more often than its written.
        \item Consistency!
        \item Know when to be inconsistent.
      \end{itemize}
\end{frame}

\begin{frame}[fragile] \frametitle{A question of good style}
  \begin{lstlisting}
    amount = 12.68
    denom = 0.05
    nCoins = round(amount/denom)
    rAmount = nCoins * denom
  \end{lstlisting}
  \pause
  \begin{block}{Style Rule \#1}
    Naming is 80\% of programming
  \end{block}
\end{frame}

\begin{frame}[fragile]
  \frametitle{Code Layout}
  \begin{itemize}
        \item Indentation
        \item Tabs or Spaces?
        \item Maximum Line Length
        \item Blank Lines
        \item Encodings
   \end{itemize}
\end{frame}

\begin{frame}{Whitespaces in Expressions}
  \begin{itemize}
        \item When to use extraneous whitespaces?
        \item When to avoid extra whitespaces?
        \item Use one statement per line
   \end{itemize}
\end{frame}

\begin{frame}{Comments}
  \begin{itemize}
        \item No comments better than contradicting comments
        \item Block comments
        \item Inline comments
   \end{itemize}
\end{frame}

\begin{frame}{Docstrings}
  \begin{itemize}
        \item When to write docstrings?
        \item Ending the docstrings
        \item One liner docstrings
   \end{itemize}
More information at PEP8: http://www.python.org/dev/peps/pep-0008/
\inctime{5}
\end{frame}

\section{Debugging}
\subsection{Errors and Exceptions}
\begin{frame}[fragile]
 \frametitle{Errors}
 \begin{lstlisting}
In []: while True print 'Hello world'
 \end{lstlisting}
\pause
  \begin{lstlisting}
  File "<stdin>", line 1, in ?
    while True print 'Hello world'
                   ^
SyntaxError: invalid syntax
\end{lstlisting}
\end{frame}

\begin{frame}[fragile]
 \frametitle{Exceptions}
 \begin{lstlisting}
In []: print spam
\end{lstlisting}
\pause
\begin{lstlisting}
Traceback (most recent call last):
  File "<stdin>", line 1, in <module>
NameError: name 'spam' is not defined
\end{lstlisting}
\end{frame}

\begin{frame}[fragile]
 \frametitle{Exceptions}
 \begin{lstlisting}
In []: 1 / 0
\end{lstlisting}
\pause
\begin{lstlisting}
Traceback (most recent call last):
  File "<stdin>", line 1, in <module>
ZeroDivisionError: integer division 
or modulo by zero
\end{lstlisting}
\end{frame}

\begin{frame}[fragile]
  \frametitle{Processing user input}
  \begin{lstlisting}
prompt = 'Enter a number(Q to quit): '

a = raw_input(prompt)

num = int(a) if a != 'Q' else 0
  \end{lstlisting}
  \emphbar{What if the user enters some other alphabet?}
\end{frame}


\begin{frame}[fragile]
  \frametitle{Handling Exceptions}
  Python provides \typ{try} and \typ{except} clause.
  \begin{lstlisting}
prompt = 'Enter a number(Q to quit): '

a = raw_input(prompt)
try:
    num = int(a)
    print num
except:
    if a == 'Q':
        print "Exiting ..."
    else:
        print "Wrong input ..."
  \end{lstlisting}  
\end{frame}

\subsection{Strategy}
\begin{frame}[fragile]
    \frametitle{Debugging effectively}
    \begin{itemize}
        \item \typ{print} based strategy
        \item Process:
    \end{itemize}
\begin{center}
\pgfimage[interpolate=true,width=5cm,height=5cm]{DebugginDiagram.png}
\end{center}
\end{frame}

\begin{frame}[fragile]
    \frametitle{Debugging effectively}
    \begin{itemize}
      \item Using \typ{\%debug} in IPython
    \end{itemize}
\end{frame}

\begin{frame}[fragile]
\frametitle{Debugging in IPython}
\small
\begin{lstlisting}
In []: import mymodule
In []: mymodule.test()
---------------------------------------------
NameError   Traceback (most recent call last)
<ipython console> in <module>()
mymodule.py in test()
      1 def test():
----> 2     print spam
NameError: global name 'spam' is not defined

In []: %debug
> mymodule.py(2)test()
      0     print spam
ipdb> 
\end{lstlisting}
\inctime{15} 
\end{frame}

\subsection{Exercise}
\begin{frame}[fragile]
\frametitle{Debugging: Exercise}
\small
\begin{lstlisting}
science = {}

for record in open('sslc1.txt'):
    fields = record.split(';')
    region_code = fields[0].strip()

    score_str = fields[6].strip()
    score = int(score_str) if score_str != 'AA' 
                           else 0

    if score > 90:
        science[region_code] += 1

pie(science.values(), labels=science.keys())
savefig('science.png')
\end{lstlisting}
\inctime{10}
\end{frame}

\begin{frame}
  \frametitle{Summary}
We have covered:
  \begin{itemize}
  \item Following and Resolving Error Messages.
  \item Exceptions.
  \item Handling exceptions
  \item Approach for Debugging.
%  \item Writting and running tests.
  \end{itemize}
\end{frame}

\end{document}

%% \begin{frame}
%%     \frametitle{Testing}
   
%%     \begin{itemize}
%%         \item Writing tests is really simple!

%%         \item Using nose.

%%         \item Example!
%%     \end{itemize}
%% \end{frame}

% \section{Test Driven Approach}
% \begin{frame}
%     \frametitle{Need for Testing!}
%    
%     \begin{itemize}
%         \item Quality
%         \item Regression
%         \item Documentation
%     \end{itemize}
%     %% \vspace*{0.25in}
%     %% \emphbar{It is to assure that section of code is working as it is supposed to work}
% \end{frame}
% 
% \begin{frame}[fragile]
%     \frametitle{Example}
%     \begin{block}{Problem Statement}
%       Write a function to check whether a given input
%       string is a palindrome.
%     \end{block}
% \end{frame}
% 
% \begin{frame}[fragile]
%     \frametitle{Function: palindrome.py}
% \begin{lstlisting}    
% def is_palindrome(input_str):
%   return input_str == input_str[::-1]
% \end{lstlisting}    
% \end{frame}
% 
% \begin{frame}[fragile]
%     \frametitle{Test for the palindrome: palindrome.py}
% \begin{lstlisting}    
% def test_function_normal_words():
%   input = "noon"
%   assert is_palindrome(input) == True
% 
% if __name__ == "main'':
%   test_function_normal_words()
% \end{lstlisting}    
% \end{frame}
% 
% \begin{frame}[fragile]
%     \frametitle{Running the tests.}
% \begin{lstlisting}    
% $ nosetests palindrome.py 
% .
% ----------------------------------------------
% Ran 1 test in 0.001s
% 
% OK
% \end{lstlisting}    
% \end{frame}
% 
% \begin{frame}[fragile]
%     \frametitle{Exercise: Including new tests.}
% \begin{lstlisting}    
% def test_function_ignore_cases_words():
%   input = "Noon"
%   assert is_palindrome(input) == True
% \end{lstlisting}
%      \vspace*{0.25in}
%      Check\\
%      \PythonCode{$ nosetests palindrome.py} \\
%      \begin{block}{Task}
%      Tweak the code to pass this test.
%      \end{block}
% \end{frame}
% 
% %\begin{frame}[fragile]
% %    \frametitle{Lets write some test!}
% %\begin{lstlisting}    
% %#for form of equation y=mx+c
% %#given m and c for two equation,
% %#finding the intersection point.
% %def intersect(m1,c1,m2,c2):
% %    x = (c2-c1)/(m1-m2)
% %    y = m1*x+c1
% %    return (x,y)
% %\end{lstlisting}
% %
% %Create a simple test for this
% %
% %function which will make it fail.
% %
% %\inctime{15} 
% %\end{frame}
% %
% 
% %% \begin{frame}[fragile]
% %%     \frametitle{Exercise}
% %%     Based on Euclid's algorithm:
% %%     \begin{center}
% %%     $gcd(a,b)=gcd(b,b\%a)$
% %%     \end{center}
% %%     gcd function can be written as:
% %%     \begin{lstlisting}
% %%     def gcd(a, b):
% %%       if a%b == 0: return b
% %%       return gcd(b, a%b)
% %%     \end{lstlisting}
% %%     \vspace*{-0.15in}
% %%     \begin{block}{Task}
% %%       \begin{itemize}
% %%       \item Write at least 
% %%         two tests for above mentioned function.
% %%       \item Write a non recursive implementation
% %%       of gcd(), and test it using already 
% %%       written tests.
% %%       \end{itemize}
% %%     \end{block}
%     
% %% \inctime{15} 
% %% \end{frame}
