%%%%%%%%%%%%%%%%%%%%%%%%%%%%%%%%%%%%%%%%%%%%%%%%%%%%%%%%%%%%%%%%%%%%%%%%%%%%%%%%
% Tutorial slides on Python.
%
% Author: FOSSEE <info at fossee  dot in>
% Copyright (c) 2005-2009, FOSSEE Team
%%%%%%%%%%%%%%%%%%%%%%%%%%%%%%%%%%%%%%%%%%%%%%%%%%%%%%%%%%%%%%%%%%%%%%%%%%%%%%%%


\documentclass[14pt,compress]{beamer}

\mode<presentation>
{
  \useoutertheme{split}
  \setbeamercovered{transparent}
}

\definecolor{darkgreen}{rgb}{0,0.5,0}

\usepackage{listings}
\lstset{language=Python,
    basicstyle=\ttfamily\bfseries,
    commentstyle=\color{red}\itshape,
  stringstyle=\color{darkgreen},
  showstringspaces=false,
  keywordstyle=\color{blue}\bfseries}

\newcommand{\kwrd}[1]{ \texttt{\textbf{\color{blue}{#1}}}  }

%%%%%%%%%%%%%%%%%%%%%%%%%%%%%%%%%%%%%%%%%%%%%%%%%%%%%%%%%%%%%%%%%%%%%%
% Macros

\newcounter{qno}
\setcounter{qno}{0}
\newcommand{\incqno}{\addtocounter{qno}{1}{Question \theqno}}

%%%%%%%%%%%%%%%%%%%%%%%%%%%%%%%%%%%%%%%%%%%%%%%%%%%%%%%%%%%%%%%%%%%%%%
% Title page
\title[Basic Python]{Python: Quiz}

\author[FOSSEE Team] {FOSSEE}

\institute[IIT Bombay] {Department of Aerospace Engineering\\IIT Bombay}
\date[] {31, October 2009\\Day 1, Quiz 1}
%%%%%%%%%%%%%%%%%%%%%%%%%%%%%%%%%%%%%%%%%%%%%%%%%%%%%%%%%%%%%%%%%%%%%%


\begin{document}

\begin{frame}
  \titlepage
\end{frame}

\begin{frame}
  \frametitle{Write your details...}
On the top right hand corner please write down the following:
  \begin{itemize}
    \item  Name:
    \item Affliation:
    \item Occupation:
  \end{itemize}
\end{frame}

\begin{frame}
\frametitle{\incqno }
  A sample line from a Comma Separated Values (CSV) file:\\
  \vspace*{0.2in}
  \emph{Rossum, Guido, 42, 56, 34, 54}\\
  \vspace*{0.2in}
  What method would you use to separate the line into fields?
\end{frame}

\begin{frame}[fragile]
\frametitle{\incqno }
  \begin{lstlisting}
  In [1]: a = [1, 2, 5, 9]
  In [2]: a[:-1]
  \end{lstlisting}
  What is the output?
\end{frame}

\begin{frame}
\frametitle{\incqno }
  How do you combine the two lists \emph{a} and \emph{b}?
\end{frame}

\begin{frame}[fragile]
\frametitle{\incqno }
  \begin{lstlisting}
  In [1]: d = {
          'a': 1,
          'b': 2
          }
  In [2]: print d['c']
  \end{lstlisting}
  What is the output?
\end{frame}

\begin{frame}[fragile]
\frametitle{\incqno }
  \begin{lstlisting}
  for x in "abcd":
      print x

  a
  b
  c
  d
  \end{lstlisting}
  How do you get the following output? 
  \begin{lstlisting}
    0 a
    1 b
    2 c
    3 d
  \end{lstlisting}
\end{frame}

\begin{frame}[fragile]
\frametitle{\incqno }
What would be the result?
\begin{lstlisting}
  In [1]: x
         array([[0, 1, 2],
                [3, 4, 5],
                [6, 7, 8]])
  In [2]: x[::-1,:]
\end{lstlisting}
\end{frame}

\begin{frame}
\frametitle{\incqno }
How to read and print each line of a file.
\end{frame}

\begin{frame}
\frametitle{\incqno }
How to get list of third column of a data file.
\end{frame}

\begin{frame}[fragile]
\frametitle{\incqno }
What is the output of:
\begin{lstlisting}
In []: x=linspace(0 , 2 * pi)
In []: plot(x, cos(x),'go')
\end{lstlisting}
\end{frame}

\begin{frame}
\frametitle{\incqno }
Draw a plot with line width 3.
\end{frame}

\begin{frame}
\frametitle{\incqno }
Setting x and y axis limits.
\end{frame}

\end{document}

