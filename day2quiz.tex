%%%%%%%%%%%%%%%%%%%%%%%%%%%%%%%%%%%%%%%%%%%%%%%%%%%%%%%%%%%%%%%%%%%%%%%%%%%%%%%%
% Quiz slides day 1 quiz 1
%
% Author: FOSSEE <info at fossee  dot in>
% Copyright (c) 2005-2009, FOSSEE Team
%%%%%%%%%%%%%%%%%%%%%%%%%%%%%%%%%%%%%%%%%%%%%%%%%%%%%%%%%%%%%%%%%%%%%%%%%%%%%%%%


\documentclass[14pt,compress]{beamer}

\mode<presentation>
{
  \useoutertheme{split}
  \setbeamercovered{transparent}
}

\definecolor{darkgreen}{rgb}{0,0.5,0}

\usepackage{listings}
\lstset{language=Python,
    basicstyle=\ttfamily\bfseries,
    commentstyle=\color{red}\itshape,
  stringstyle=\color{darkgreen},
  showstringspaces=false,
  keywordstyle=\color{blue}\bfseries}

\newcommand{\kwrd}[1]{ \texttt{\textbf{\color{blue}{#1}}}  }

%%%%%%%%%%%%%%%%%%%%%%%%%%%%%%%%%%%%%%%%%%%%%%%%%%%%%%%%%%%%%%%%%%%%%%
% Macros

\newcounter{qno}
\setcounter{qno}{0}
\newcommand{\incqno}{\addtocounter{qno}{1}{Question \theqno}}

%%%%%%%%%%%%%%%%%%%%%%%%%%%%%%%%%%%%%%%%%%%%%%%%%%%%%%%%%%%%%%%%%%%%%%
% Title page
\title[Basic Python]{Python for science and engineering: Day 2, Quiz 1}

\author[FOSSEE Team] {FOSSEE}

\institute[IIT Bombay] {Department of Aerospace Engineering\\IIT Bombay}
\date[] {\today \\ Day 2, Quiz 1}
%%%%%%%%%%%%%%%%%%%%%%%%%%%%%%%%%%%%%%%%%%%%%%%%%%%%%%%%%%%%%%%%%%%%%%


\begin{document}

\begin{frame}
  \titlepage
\end{frame}

\begin{frame}
  \frametitle{Write your details...}
On the top right hand corner please write down the following:
  \begin{itemize}
    \item Name:
    \item University/College/Company:
    \item Student/Teacher/Professional:
  \end{itemize}
\end{frame}


\begin{frame}
\frametitle{\incqno }
  What is the largest integer value that can be represented natively by Python?
\end{frame}

\begin{frame}
\frametitle{\incqno }
  What is the result of 17.0 / 2?
\end{frame}

\begin{frame}
\frametitle{\incqno }
  Which of the following is not a type in Python?
  \begin{enumerate}
    \item int
    \item float
    \item char
    \item string
  \end{enumerate}
\end{frame}

\begin{frame}
\frametitle{\incqno }
How do you create a complex number with real part 2 and imaginary part
0.5.
\end{frame}

\begin{frame}
\frametitle{\incqno }
  What is the difference between \kwrd{print} \emph{x} and \kwrd{print} \emph{x,} ?
\end{frame}

\begin{frame}
\frametitle{\incqno }
  What does '*' * 40 produce?
\end{frame}

\begin{frame}[fragile]
\frametitle{\incqno }
    What is the output of:
    \begin{lstlisting}
In []: ', '.join(['a', 'b', 'c'])
    \end{lstlisting}
\end{frame}

\begin{frame}
    \frametitle{\incqno}
  How do you find the presence of an element \emph{x} in the list \emph{a}?
\end{frame}

\begin{frame}[fragile]
    \frametitle{\incqno}
  \begin{lstlisting}
In []: set([1, 2, 8, 2, 13, 8, 9])
  \end{lstlisting}
  What is the output?
\end{frame}

\begin{frame}[fragile]
    \frametitle{\incqno}
  \begin{lstlisting}
In []: 47 % 3 
  \end{lstlisting}
  What is the output?
\end{frame}

\begin{frame}[fragile]
    \frametitle{\incqno}
  \begin{lstlisting}
In []: a = 12
In []: a *= 1+1
  \end{lstlisting}
  What is the value of a?
\end{frame}

\begin{frame}[fragile]
    \frametitle{\incqno}
  \begin{lstlisting}
In []: a = {'a': 1, 'b': 2} 
In []: a['a'] = 10
In []: print a
  \end{lstlisting}
  What is the output?
\end{frame}

\begin{frame}[fragile]
    \frametitle{\incqno}
  \begin{lstlisting}
In []: for i in range(3, 10, 2):
  ...:     print i
  \end{lstlisting}
  What is the output?
\end{frame}

\begin{frame}[fragile]
    \frametitle{\incqno}
  \begin{lstlisting}
In []: a = [1, 2, 3] 
In []: a.extend([5, 6])
  \end{lstlisting}
  What is the value of a?
\end{frame}

\begin{frame}[fragile]
    \frametitle{\incqno}
  \begin{lstlisting}
In []: a = (1, 2, 3)
In []: a[1] = 10
  \end{lstlisting}
  What is the result?
\end{frame}

\begin{frame}[fragile]
    \frametitle{\incqno}
  \begin{lstlisting}
def func(x, y=10):
    print x+1, y+10

func(1)

  \end{lstlisting}

  What is the output?
\end{frame}

\begin{frame}
    \frametitle{\incqno}
  How many items can a function return?
\end{frame}

%% \begin{frame}[fragile]
%%     \frametitle{\incqno}
%%     Consider a module called \lstinline+gcd.py+ looking like this:
%%     \begin{lstlisting}
%% def gcd(a, b):
%%    ...

%% if __name__ == '__main__':
%%     print gcd(10, 25)
%%     \end{lstlisting}
%% If this module is imported, will it print the gcd of 10 and 25?
%% \end{frame}

%% \begin{frame}[fragile]
%%     \frametitle{\incqno}
%%   \begin{lstlisting}
%% In [1]: print hello 
%%   \end{lstlisting}
%%   Exactly what exception will you get if you run this on a fresh
%%   interpreter?
%% \end{frame}

\end{document}

